\chapter{C++ Object-Oriented Programming}

The \cpppl\ is a multi-paradigm programming language. The ability to support the object-oriented programming,
is incorporated via classes. 

A class represents a new type in the \cpppl. Each class may have data in the 
form of fields, and behavior in the form of methods. Two types of methods are special for
C++ - constructors and destructors, they have special meaning and syntax.

Encapsulation is enabled via governing access permissions to fields and methods of a class.
The access control is governed with the creation of \cc{public}, \cc{protected} and \cc{private}
class sections.

Inheritance is implemented in C++ via allowing for each class to have one, or even many base classes.
Inheritance from a base class is performed under a certain access control modifier. There is no pure notion 
of an interface, but rather abstract classes are introduced.

Polymorphism is implemented via pointer-to-class type compatibility over inheritance-connected classes.

The implementation of these C++ features in a projectional editor environment is discussed in the
following chapters.


\section{Class declaration}
\label{section:classes}


% Enforcement of one section and only one.
% Enforcement of sections order
% Constructor markup
% Copy constructor and constructor generation


The \cpppl\ is a mature language, with long traditions, and high flexibility, \cite{alexandrescu} can serve as an example. 
Although the language is also famous for being complex. It will not be possible to simplify language, removing features from 
it, which will restrict the language use. In this work I try to research, how the editor can be more supportive for the user,
to eliminate usual mistakes made while programming, as well as provide help in structuring the code.


\subsection{Visibility Sections}
\label{visibilitysections}


Instead of declaring visibility type for individual class members, visibility sections are created
in C++. 

The sections can be opened with a string \cc{private:}, \cc{protected:} or \cc{public:}
within a class declaration, and closed when another section is opened or when the class declaration
ends. This allows the user to open and close the same section multiple times and declare sections
without any particular order.

Various coding guidelines (\cite{cppclasslayout}, \cite{googlecppstyle}) exist to enforce some 
restrictions on the visibility sections.

In particular, the sections are allowed to be opened only once. This ensures, the
reader of the code will see interface of the class (public section) in one place,
``contents'' of the class (private section) in one place, and opportunities to access members in
the inheriting class (protected section) in one place, without the need to scroll through through
the whole class declaration.

Another typical requirement in coding standards, is the order of the section. Usually the public 
section is required to be first, for the class users to see immediately the public interface,
the class provides.

\ms{webpage}{Sample class type declaration}


The \fig{webpage}, shows an example class declaration implemented in the projectional editor.
The class concept has the visibility sections as children. Each section is given a separate role,
and can appear 0 or 1 times. The editor for the class concept orders the visibility sections so,
that the public section always comes first, followed by the private and protected sections.

The creation of a section is made with the use of so called \emph{intention} in MPS. The user
uses \emph{Alt+Enter} combination on the class declaration to create visibility sections.
It should be more practical and fast for the user, compared to typing the keyword, colon and
indenting the result.

% Limitation
A question arises on how to support another way to represent a class, so that it will reflect
requirements from a different coding standard. And as a way to resolve it a definition of 
another editor for a class concept can be offered, Unfortunately, the current version of \jbmps\ 2.5
does not support a definition of multiple editors for the same concepts. This limitation however
is addressed in the newer 3.0 version.


\subsection{Constructors}
\label{classconstructors}

Constructors are special methods of a class, used to construct the class instances. 

Constructors have special syntax and no return type, being similar to class methods. Additional 
value, however, the constructors gain, when participating in type transformations. Namely, when
a constructor of a class B exist from a type T, instances of the type T can be used whenever
the B class instance is required. The constructor will be \emph{implicitly} called and a temporary object
of class B is going to be created as a mediator. 

Thus constructors extend the type system. Since this type extension can not be easily observed, it is 
highly possible to get various \emph{run-time} errors or unexpected behavior. 

\cpp{Example of implicit constructor error}{implicit}

The \rl{implicit} demonstrates a simplified use case where the function \cc{print()} is invoked on \cc{int} without 
any compiler error, and the resulting behavior is unexpected.

To avoid similar, it is possible to deprecate participation of a constructor in type conversions, adding a word \cc{explicit} to the constructor
declaration.

The described problematic motivated the following decision. When a new constructor is created, it is by default declared to be explicit,
the user must intentionally change it to get the type conversion behavior. Such behavior is safer by default.

\subsection{Copying}
\label{classcopying}

In the \cpppl\ the programmer controls memory allocation fully on his/her own. This affect the way of copying the object.
In C++ a programmer should define two methods for a class: a copy constructor and an assignment operator.
These two methods work when assignment like \cc{a = b} happens, when instances of a class are passed by value to a function, 
when one object is initialized with a value of another object and so on.

C++ serves here sometimes dangerously generating default copy constructor and assignment operator, which by default represent
a bitwise cloning of an object. This can lead to problems.

\cpp{Need in custom copy constructor}{copydeath}

The \rl{copydeath} demonstrates a program which crashes upon execution as destructors of \cc{a} and \cc{b} are deallocating 
memory with the same address, after default copy-constructor copies the address from \cc{a} into \cc{b}.

To avoid the described problem, the programmer has to either define proper copy-constructor or forbid copying of the objects of
the class. The same applies to the assignment operator. Many standards require the two functions to be implemented in sync, \cite{ooocpp}.
This can be performed be implementing the assignment operator first and reusing it in the copy-constructor. To simplify the first,
specialized macros exist, for example \cc{DISALLOW\_COPY\_AND\_ASSIGN} or \cc{Q\_DISABLE\_COPY}, \cite{googlecppstyle}, \cite{qobjref}.

The use of macros in C++ appears often in similar cases, in order to perform some language-engineering tasks to add the missing
features to the language. Macros bear pure textual nature, and are processed by the pre-processor. Some negative effects may 
come out: need to pre-process reduces the speed of compilation and hides the resulting code from a programmer, macros lead
to error prone programming, as no type checks are possible, macros make code less analyzable by automatic analyzers.

\ms{copyable}{Hinting about copyable and assignable class properties}


The projectional editor allows for another solution, different from macros.  In order to provide some support for the 
programmer regarding the copying issue, the projectional C++ implementation hints on the class declaration its 
assignable and copyable properties, \fig{copyable}, and generates by default the declarations of copy-constructor and assignment operator.

The copy constructor and the assignment operator are detected automatically. Two intentions are provided on the Class concept to
forbid or allow copying. The forbidding intention imitate the macros mentioned above, but displaying and explaining the 
implementation to the user\fig{notcopyable}. The implementation consists of moving the declarations of two functions to the private section of
the class. Implementation is not required for such functions.

\ms{notcopyable}{Class made not copyable by the \emph{Forbid copying} intention}

The \jbmps\ supports the implementation by providing read-only model accesses, special parts of editor concept, 
by which the hinting is implemented. The intentions allowed manipulations on a class, which made it possible to 
automate the allowing or forbidding of copying/assignment, making the implementation clear to the user, without the 
use of macros. The whole work to forbid copying and assignment contains of a call to an intention, one key-stroke.
There is no need to include a header file with macros, and look up the documentation for them (symmetric macro requires
exactly one parameter - the class name, and it has to go in the private class section).

\section{Encapsulation and Inheritance}

% what to support in C++

Encapsulation and inheritance are considered here together, because from the language-engineering point of view, 
they just decide the access for class members. In other words, the projectional C++ implementation has to track
encapsulation and inheritance related definitions and provide access to the class members accordingly.

% Problem

\subsection{Various Cases of Access Control}

In the \cpppl\ exists a number of ways to govern access control to class members. Before discussing 
the implementation of them in a projectional C++, I briefly review them with an example.

All members, a class has, are either declared in the class, or inherited from its base classes. 

The members can be accessed in a number of different locations in the code, which differ by the level of access they have
to the class members. Among these locations are the class methods, friend functions, belonging to the class and
external to the class code. 

Each member can be declared with a certain visibility/access type, and the inheritance
of the member can happen with one of the three inheritance modifiers. 

The access depends on the object, on which the member is requested, and on its type as well.

\ms{inhdecl}{Declaration of two classes with a friend function}

The \fig{inhdecl} shows a declaration of two classes. The class \cc{A} has all three
public, protected and private fields. A function \cc{compare()} is declared to be a
friend function. The visibility plays no role for the friend function declarations, 
that is why a decision was made to create a special section for friends, it is not 
generated anyhow in the resulting C++ textual code.

The class \cc{B} in the \fig{inhdecl} is inheriting publicly from A, which means,
that public members of \cc{A} remain being public in \cc{B}. The class \cc{B} declares
a copy constructor.

Such declaration can be utilized as shown in the figure \fig{inhusage}.

\ms{inhusage}{Visibility resolution}

In the copy-constructor the visibility resolution happens after \cc{this} pointer
and after the \cc{original} object. Arrow expression and dot expression are used for
this. The first and second lines are making use of public and protected fields of 
the base class \cc{A}. The use of the private field is however not possible, since
private fields of a class are only accessible to methods of the same class and
friend functions. 

It is even not possible to input the not-allowed member, as the projectional editor
does not bind the text to anything, and it remains red, invalid, not usable further.


The \cc{compare()} function is declared in advance (\fig{inhdecl}) to be a friend 
function of the class \cc{A}. Thus, it is not a problem for this function to access
even private fields of \cc{A}, for comparison purposes in this example case.

The function \cc{external()} is not related anyhow to classes declared. Thus,
it represent ``external'' from the class point of view code. Only public members
are accessible, but not protected or private. The attempt to input them, simply
fails, they are highlighted red, and not bound to anything.


\subsection{Expressions to Address Class Members}

% MPS support

Members are usually accessed relatively to some object. The object can be designated as an expression of type class or a pointer to class,
in particular, \cc{this} expression. The resulting access represent nothing else, but expression itself. 
One of the greatest ideas in \jbmps\ to allow extensibility is concept inheritance. Once a need to create a new concept arises, serving as a 
concept known before, the new concept has to inherit from the existing one, and this is almost all what has to be done.
Thus inheriting the OoDotOrArrowExpression concept from the Expression concept, we get the ability to use the expression, designed for member 
access, wherever an expression can be used. 

The abstract concept OoDotOrArrowExpression serves as a parent for OoDotExpression and OoArrowExpression. The commonality between the two, is
that an object is accessed in the left part, and a member is selected in the right part, as well as the way to decide the access to members.
The access is defined then, which left part is going to be possible in such expressions.


% Limitation
Within the class methods it is also possible in C++ to address class members as local variables. In the projectional implementation
described, it is not possible. Instead, \cc{this} expression has to be used. It makes typing a little bit slower, but allows to
easily distinguish between members of the class and other variables or functions.



\section{Polymorphism}

There are several ways to achieve polymorphic behavior in C++. The purists of the language differentiate 
between the polymorphism based on virtual functions or on templates. More general opinion can include
in the notion of the polymorphism also functions overloading and operator overloading, also called 
\emph{ad hoc} polymorphism. Occasionally various operations with \cc{void *} are also classified as 
the polymorphic programming.

In this section I am writing only about the class-related virtual functions polymorphism.

\subsection{Virtual Functions Polymorphism in C++}
\label{cpppolydefs}

Starting the description of the projectional implementation, I find it good to spend some time describing
the way, the polymorphism is implemented in the \cpppl\ itself.

Dislike many other popular object-oriented programming languages in the C++ there is no pure notion of
an interface. Instead, a base class is used as an interface declaration for its descendants. Functions
designated to be a part of the interface must be declared \cc{virtual} and they can be overloaded in
the subclasses.

The virtual functions in the base-interface class can be implemented as well, providing some ``default''
implementation. Otherwise they are left \emph{pure virtual}, meaning that no implementation is provided
and the pointer to the function in the table of pointers to virtual functions is zeroed. The syntax
for pure virtual functions is rather not obvious, when not knowing about the zero pointer semantics,
and a bit cumbersome requiring to type one reserved word, one punctuation sign and one number to 
express simple fact of pure virtuality or, simply, absence of implementation, \rl{purevirtual}.

\cpp{Pure Virtual Function Syntax Example}{purevirtual}

Such approach is more flexible compared to languages with the notion of interface directly introduced.
In C++ it is possible to create partially implemented base classes, what can not be done when implementing
and interface in Java or C\#, where the approach is ``all or none'' regarding the implementation of interface
functions.

In order to implement a declared in a base class function the descendant must declare and implement
a virtual function, matching the full (with return type) signature of the declared function, \rl{override}. The connection 
to the ``interface'' function declaration stays subtle, and it is not immediately clear, whether the 
declared implementation is an override or a new independent declaration. This knowledge affects the 
changing process greatly, as the override should change from the interface, together with all the implementation.
The absence of a clear override syntax I call here ``override syntax absence''.

Whenever a class and all of its ancestors does not provide an implementation of a certain virtual function,
created as a pure virtual in the declaring class, the class is called \emph{abstract}. It is not possible
to construct instances of abstract classes. C++ however does not have any special syntax for abstract classes,  \rl{purevirtual}.
The programmer usually has to be aware (from documentation, implementation, or, the worst case, compilation
errors) whether a given class is abstract. I will call this phenomenon as ``abstract class syntax absence''.

When the overriding a function interface is a active action of the programmer, and is initiated by the 
programmer, the abstract property of the class, oppositely, can be deduced from the analysis on 
the base classes automatically, by editor.

In order to use an interface declared in some class, the using code has to get a pointer to instance
of any implementing the interface descendant class instance. Thus typing system has to allow a pointer
to the descendant to be treated in the way as a pointer to the base would be. The same should hold 
normally for the reference types, but it is not used very often on practice, and omitted in the implementation.
As this typing rule represents the core of polymorphic behavior, I will start from it, describing further the
projectional C++ implementation.

\subsection{Pointer to Class Special Typing}

The problem solved here is enabling the usage of pointers to descendants instead of pointers to ancestors,
\rl{polycall}.

\cpp{Example Usage of a Pointer to Descendant Class}{polycall}

The pointer type is implemented in a separate language in mbeddr. Following the \rsgoal{3} the C++ implementation
needs to add the typing rules for pointer \emph{to classes} without changing the pointer mbeddr language, where the 
typing system for pointer type is defined.

In the case when a type of a pointer to a base class is expected, a pointer to a subclass should also be accepted.
The type system of the pointer language will try to check the compatibility of the two pointer types. It will
fail to do so, as the mbeddr languages are not aware of the C++ extensions by design.

When none of the existing type system rules can resolve the given situation, the so called replacement
rule can be invoked in MPS. The C++ extension provides a replacement rule, which checks, whether a
class pointed to, is a passing descendant of the class required by an expression, where the pointer was
used.

In the \rpart{extensibility} I discuss more on the approach taken here, its limitations and 
the potential ways to overcome them.

\subsection{Overriding a Virtual Function}

The \rl{override} demonstrates, how overriding of a virtual function happens in C++.
The function \cc{getArea()} is initially defined in the class \cc{Shape} and is then overridden
in the class \cc{Circle}. The \rl{override} demonstrates as well the ``override syntax absence'' in C++.

\cpp{Example of an Overridden Function - Text Code}{override}

In the \fig{override} the same example is demonstrated, but in the projectional C++ implementation.
Indeed, the \rl{override} is the generation result part from the demonstrated projectional sample.

When a method declaration is created, it can be set to be an override of a method in a base class.
Only when an empty method declaration is created, it can be set to be an override, so that the only
thing which stays, is to pick a method from a base class to be overridden. The override is automatically
named, the parameters and the return type are set accordingly, the \cc{virtual} property is immediately
set. 

After the override has been linked to the overridden method, the projectional editor checks, if 
the override full signature stays precisely the same as the one of the overridden method. 
The overridden method is shown next to the override declaration \fig{override}, which 
compensates on the override syntax absence.

\ms{override}{Example of an Overridden Function - Projection}

The additional syntax on the projectional view, should not confuse the reader. It is only 
present in the projectional editor, and, when the AST is generated into code, the regular
C++ syntax is achieved. However in this case the AST stores \emph{more} than needed to 
generate and compile the C++ code. This subsection demonstrates, why it can be useful.



\subsection{Pure Virtual Functions}

In the example above one improvement to the code can be made. As the \cc{getArea()} for a random shape can not 
be determined, it makes sense to make the \cc{getArea()} function pure virtual. A pure virtual function has no 
implementation in the declaring class, and serves only overriding purpose. Semantically in C++ it sets the 
pointer in \cc{vptr} to 0 and thus has reflecting this syntax, see the similar example in \rl{purevirtual}.

This syntax can be seen as not obvious, as it reflect more the under-the-hood implementation of the mechanism,
rather then the original programmer intention to built a basis for an overrides chain. 

Additionally, as discussed above, declaring a pure virtual function requires a significant amount of the syntactical
overhead.

In the projectional C++ implementation, one intention is reserved to make a function pure virtual. The intention automatically
sets the needed pure virtual and virtual flags of the function declaration, and the projection changes. That is why out of the 
statement \cc{virtual double getArea() = 0;} the programmer has to input only the name \cc{getArea}, pick the type \cc{double}, and
toggle the pure virtual intention for the declaration. 

The result of the intention work in projection can be seen in the \fig{purevirtual}.

\ms{purevirtual}{Example of a Pure Virtual Method - Projection}

The word \cc{pure} is added by the projection editor. This makes the reading of the code easy and natural.
The \cc{ = 0} part is preserved for the C++ programmer, used to this syntax. And, as in many other cases, 
the semicolon is omitted as it is nothing but syntactic help to the compiler, which does not have to appear 
in the projection.

\subsection{Abstract Classes}
\label{abstractclasses}

If any of the pure virtual functions in the inheritance chain leading to a class is not implemented by this chain, 
or inside the class itself, the class is called abstract. It is not possible to construct an instance of an
abstract class, as such classes are partially implemented. The programmer has to know which classes are abstract,
and are intended for inheritance and further implementation only.

Above I discuss the absence of the abstract class syntax in C++. It leads to the need for the programmer to 
determine somehow him/herself if a given class is abstract. 

And editor can however perform an analysis and determine if the class is abstract. In the case of the projectional
editor, this analysis is especially computationally efficient, as the AST is readily available, and the quick 
analysis can be performed by a simple recursive algorithm on the inheritance chains.

After an abstract class is determined, it is possible to modify the editor representation for it, and show that it is 
abstract, \fig{abstract}.

In this example a typical class hierarchy is created to support user interface programming. A user of this API, when 
searching for a button, could try using the \cc{Button} class, which is designed to be abstract, and serve as a 
base for the further implementation, e.g. \cc{PushButton} in the example, or check boxes, radio buttons and similar,
later.

\ms{abstract}{Determining Abstract Classes}

The projectional editor, however, checks on the fly, if a certain class is abstract, adding a special \cc{abstract} word 
in front of its declaration. This makes the reading easier, and allows for quicker understanding of the code. 

Additionally, the creation and usage of abstract class instances is checked, and forbidden by type analysis. This is 
described separately in the \rchap{advanced}.

% To be used later when writing about type conversion.

% Constructors among the special meaning of creating 
% instances of the type class, are also participating in the type conversion mechanisms.
% Namely, if there is a constructor a class of \cc{B} from a class \cc{A}: \cc{B(A obj)} or 
% similar, any object of the class \cc{A} can participate in places, where an object of the 
% class \cc{B} is requested originally, as implicit type conversion takes place, invoking the 
% \cc{B(A obj)} constructor.


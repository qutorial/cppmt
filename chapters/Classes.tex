\chapter{C++ Object-Oriented Programming}

The \cpppl\ is a multi-paradigm programming language. The ability to support the object-oriented programming,
is incorporated via classes. 

A class represents a new type in the \cpppl. Each class may have data in the 
form of fields, and behavior in the form of methods. Two types of methods are special for
C++ - constructors and destructors, they have special meaning and syntax.

Encapsulation is enabled via governing access permissions to fields and methods of a class.
The access control is governed with the creation of \cc{public}, \cc{protected} and \cc{private}
class sections.

Inheritance is implemented in C++ via allowing for each class to have one, or even many base classes.
Inheritance from a base class is performed under a certain access control modifier. There is no pure notion 
of an interface, but rather abstract classes are introduced.

Polymorphism is implemented via pointer-to-class type compatibility over inheritance-connected classes.

The implementation of these C++ features in a projectional editor environment is discussed in the
following chapters.


\section{Classes}
\label{section:classes}


% To be used later when discussing inheritance

% Although
% the three inheritance modifiers are named in the same way as in-class access modifiers


% To be used later when writing about type conversion.

% Constructors among the special meaning of creating 
% instances of the type class, are also participating in the type conversion mechanisms.
% Namely, if there is a constructor a class of \cc{B} from a class \cc{A}: \cc{B(A obj)} or 
% similar, any object of the class \cc{A} can participate in places, where an object of the 
% class \cc{B} is requested originally, as implicit type conversion takes place, invoking the 
% \cc{B(A obj)} constructor.

\section{Encapsulation and Inheritance}

\section{Polymorphism}


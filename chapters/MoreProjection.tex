\chapter{Generalized Principles of the Projectional Approach}
\label{chapter:genprinciples}

In this chapter I formulate some of the general principles, which can be taken into
account when designing new languages in the projectional editors.

\section{Targeting Semantics}
%No matter what you generate to, you should target semantics - way to cover the abstraction gap

\section{Store More Information}
% More information can and SHOULD be stored than is needed for generation.
% Example is the overridden method link in the override

% Disadvantage - native code compatibility

\section{Configuration as a Part of Source Code}
% Configuration (as naming conventions) is nice to store with code! 
% Usually editor preferences are not shared like in eclipse, but this can enforce standards
% Another example would be the build configuration indeed.

%TODO More?


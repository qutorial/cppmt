\chapter{Advanced Editor Functionality}
\lchap{advanced}

As the projectional editor work directly with an  AST, it is possible to provide some programming on the AST
to improve the user experience. I call this additional programming as advanced editor functionality, and 
discuss it in this chapter.

\section{Renaming Refactoring}
Directly from the nature of the projectional editor, without any additional effort, comes a feature to 
perform the renaming refactorings.

If a node of an AST gets to be referenced somewhere else, it is referenced by the use of its unique
identifier. The name of the node is a property of the node, which is not playing any role in the
referencing the node from somewhere else. Thus renaming, dislike the way it is performed in text
editors, does not involve replacing of the name all around the code. Instead, just the name property 
of a node is changed.

As an example - renaming a class or a method would mean just changing its name where it is declared first.

As a disadvantage here one can think of moving a code from one place to another. For example, moving 
a method from one class to another. The member fields of the source class, even if present in the destination
class, will not immediately fit into the copied new code, as it is still referencing not available anymore
identifiers of the source class member fields. This effect is also mentioned in the \rchap{comparison}.

% Primitive renaming refactorings out of the box

\section{Getter and Setter Generation}
\label{section:getterandsetter}

In order to provide access to encapsulated class properties, usually expressed as member fields in C++, 
two access functions are defined, known as a getter and a setter. The getter is used to read the property,
and the setter is used to set the property to a new value, after checking the validity of the new value.

The job of declaring and prototyping the two functions can be automated with an MPS intention, \fig{getterandsetterintention}.

\msnozoom{getterandsetterintention}{Calling the Generation of Getter and Setter}

The result of the intention work is, as expected, two methods declared and prototyped, \fig{getterandsetter}.

\msnozoom{getterandsetter}{Getter and Setter Generated}

The way the editor names the getter and setter can be controlled through Naming Conventions concept, which
is discussed in the \rsec{namingconventions}.

The getter and setter are declared in the public section of the class, which is automatically created, if
not found there already.

The getter is rather simple, it just returns the value of the member field.

The setter is somewhat more complicated. Firstly, it designed to return \cc{boolean} by default. This is made
to remind the programmer, to include checking of the value and return \cc{false}. Secondly, the parameter of the
getter is typed appropriately. As C++ by itself and the \rsgoal{1} imply the performance maximization, the type
of the parameter for the setter depends on the member field type. And when the type represents a composite
structure, like a class, it is passed by a constant reference, instead of value, \fig{compositesetter}.

\ms{compositesetter}{Setter Works with a Constant Reference for Classes}

Passing a composite parameter as a value involves an overhead of allocating the necessary memory and 
then copying the contents in the newly allocated memory. To access the parameter in both cases pointer
arithmetic is going to follow.

% Getter and setter generation
% Getter and setter do not keep name synch for compatibility with of interfaces with referencing text code


\section{Naming Conventions}
\label{section:namingconventions}

In the C++ development projects some code writing conventions are usually agreed upon. For example, 
in Google coding style guide for C++ \cite{googlecppstyle} the member fields must end up with ``\_'' sign.
For the consistency and uniformity purposes each project has to have some agreements on code  composing.
They can include code formatting rules and naming rules. 

To some extent the formatting conventions are fixed
by the way the projectional editor shows the AST, see, for example, the Section \ref{visibilitysections}.

The naming rules, however, should be additionally controlled. The naming rules accepted in a project I call 
naming conventions here.

In order to perform this a new concept has been introduced, called C++ Naming Conventions, \fig{namingconventions}.

\ms{namingconventions}{C++ Naming Conventions Concept}

In the naming conventions one can give a standard prefixes for getters and setters, which are used during the code 
generation for them, see the \rsec{getterandsetter}. The argument prefix is also used in the generation, to name the
setter argument.

The member prefix is used on the member fields to check their naming. It possible to change the prefix and the editor will control
each of the member field names to comply.

The naming conventions concept was created to demonstrate some new possibilities, which can be added to the projectional C++.
Few remarks are needed to explain potential changes to the concept and its difference from the similar functionality taken
in the conventional text editors.

First, the marking of special variables or object, keywords or type names, can be performed be changing editors for them. The special
naming will not be needed then. Such way would mean however no opportunity to introduce a project-dependent marking, as it starts to be
common for all projects, being a part of the projectional C++ implementation.

Second, the naming checks can be generalized to use compliance to regular expressions, instead of just prefixes. As an advantage comes the
ability to adopt a much broader set of different naming conventions. Disadvantageous is the increase
of the computational load in the projectional editor, this is discussed also in the \rchap{analysisandcomplexity}.

Third, the naming conventions in projectional C++ is just a concept in an MPS model. It is no different from a C++ Implementation Module
or Build Configuration. Dislike many conventional editors, which store such settings as a workspace, project or IDE configuration, the
projectional C++ makes it a part of the program code itself. It is advantageous, since all the programmers in a team have to follow the 
same naming conventions set up once in a project, and no environment (re-)configuration is needed to work on the project. The porjectional
approach is also discussed and compared to the conventional one in the \rchap{comparison}.
% Naming conventions
% Improvement idea: Regular Expressions

% Check for the method implemented

% Check for abstract classes construction

%TODO Deafault constructable and arrays! Require constructor by default

%TODO Check for Virtual Destructor 

% Describe Abstract class check Abstract class is not created
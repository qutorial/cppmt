\declsec{Templates}{templates}

\cppproblem


Templates represent a powerful tool in the \cpppl. 

In fact, templates, especially used together with a preprocessor, represent a language engineering tool, which allows for extending C++ with additional 
constructions, not originally present in the language, \cite{alexandrescumeta}. In this regard an approach 
taken in projectional editing is an alternative, as projectional language modularity is considered to be 
a basis for language extending, \ref{modularity}.

However, templates have to be supported to some extent in the \pcpp, as \rg{stl} is based on templates entirely. Without 
supporting templates, it would not be possible to provide a usual for a C++ programmer standard library. Mostly
templates are used to abstract over some type in \rg{stl}.

Templates bare a pure textual structure in C++. A template code is not even syntactically checked before instantiation,
i.e. template code is not even parsed before an instantiation happens.

When instantiating a template, the assumptions taken in the template code on a template parameter are checked against 
a concrete template parameter, instantiating the template. These assumptions are implicit in C++. The template parameter
is assumed to be capable of everything, which is possible in C++: participation in all types of expressions and statements
is assumed, in each role.

The described features of template code can present a source of various kinds of errors,
for instance:
\begin{itemize}
 \item A syntactic error found on the time, the template code is used first, and not before, when the code is created
 \item The template code receives precise semantic meaning, only during instantiation, before the instantiation it is not known, and it can be understood wrongly,
 when, for example, the template code is called somewhere, not being instantiated.
 \item Assumptions put on the template parameter may conflict, when one template parameter is to be used with several template code fragments
 \item The assumptions are hidden from the programmer being implicit requirements in the template code
\end{itemize}

In projectional editing, constraints are required, as they define, what can be constructed in principle, \ref{mpsconstraints}.
This contradicts with the ``unconstrained'' nature of template parameters. For instance, when a template parameter represents a 
class-like type, any method can be called on an object of the type. However, in the projectional editor, only methods, previously
defined, can be called on a class type. This demonstrates an example of contradiction in nature of templates in C++ and the projectional
editing with constraints in general.

\pcppsolution

As a way to support templates in \pcpp\ we introduce a term of C++ concepts. This term is not to be confused with \rgp{concept} in 
projectional editing.

A C++ concept explicitly describes an assumptions put on a template parameter. A template code is required to declare, which 
C++ concept a template parameter belongs to, before the parameter can be used. The C++ concept defines the way a template 
parameter could be used in the template code.

\gr{templates}{Template Code Sample in \pcpp}{0.1}

The \fig{templates} demonstrates how a template code can be composed with the use of C++ concepts.

At first \cc{Comparable} concept is declared. It puts a requirement on a type to have a function 
\cc{compare()} in its public section.

Next, a class \cc{NumberWrapper} is declared to \cc{realise} the \cc{Comparable} concept. The
editor will check the class upon the requirement.

A template class \cc{OrderedList} declares a template parameter \cc{T} and specifies that it 
satisfies the \cc{Comparable} concept. This makes an object of type \cc{T} usable as if it was 
satisfying the requirements as declared by \cc{Comparable}. This is demonstrated in the \cc{OrderedList::compare()}
function.

When instantiating a template, it is checked if either the parameter given realizes the needed 
C++ concept, or, more flexibly, if it satisfies the C++ concept requirements, without declaring
the realization explicitly.

The support for templates by the \pcpp\ is, of-course, limited, when compared to traditional C++ facilities.
It is evaluated in the \rsec{templatesevaluation}.

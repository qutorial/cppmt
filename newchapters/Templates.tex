\declsec{Templates}{templates}

\cppproblem


Templates represent a powerful tool in the \cpppl. 
In fact, templates, especially used together with a preprocessor, represent a language engineering tool, which allows
for extending C++ with additional 
constructions, not originally present in the language, \cite{alexandrescu}. In this regard an approach 
taken in the projectional editing is an alternative, as the projectional language modularity is considered to be 
a basis for the language extending, see the Section \ref{modularity}.

However, templates have to be supported to some extent in \pcpp, as \rg{stl} is based on templates entirely. Without 
supporting templates, it would not be possible to provide a usual for a C++ programmer standard library. Mostly
templates are used to abstract over some type in \rg{stl}.

\paragraph{Several Disadvantages of Templates in C++}  Templates bear a pure textual structure in C++. A template code is not even syntactically checked before instantiation,
i.e. template code is not even parsed before it gets explicitly used, or instantiated, in a compiled, working, code.

When instantiating a template,  assumptions, taken in the template code on a template parameter, are checked against 
a concrete template parameter. These assumptions are implicit in C++. The template parameter
is assumed to be capable of everything, which is possible in C++. Participation in all types of expressions and statements
is assumed, in each role.

The described features of a template code can present a source of various kinds of errors,
for instance:
\begin{itemize}
 \item A syntactic error found on the time, the template code is used for an instantiation first, 
 and not before, when the code is created
 
 \item A template code receives precise semantic meaning, only during instantiation.
 For a user to understand, what the template code really does, the user has to know the implementation of the template, namely,
 how exactly the real parameters are going to be used.
 
 \item Assumptions, put on a template parameter, may conflict, when one template parameter is to be used with several 
 template code fragments, putting different assumptions on the parameter.
 
 \item The assumptions are hidden from a programmer being implicit requirements in the template code. Staying implicit, they can
 be not fully observed, or understood wrongly.
 
\end{itemize}

\vspace{7mm}

\paragraph{The Conflict brought by the Textual Nature of Templates}

In the projectional editing, constraints are required, as they define, what can be constructed in principle, see the Section \ref{mpsconstraints}.
In other words, before using a certain object to construct with it nodes on an \rg{ast}, in the projectional editor it has to be 
precisely known, what this object is capable of doing, how it can be used. This is defined through the special ``scoping constraints'',
see the Section \ref{mpsconstraints}. For example, when an instance of a class is used, it has to be
known \emph{in advance}, which methods the class has, in order to call them on the instance. When using, for instance, a template 
parameter type, it is not known in C++, what are its capabilities. They are being determined later, as the mentioned above
implicit assumptions, put on to the type, when it was used by template code. This example represents the ``unconstrained'' 
nature of template parameters in C++. It is a contradiction in nature of templates in C++ and the projectional
editing with constraints in general. A special modification to C++ language is presented below as a way to resolve the 
formulated problem.

\pcppsolution

As a way to support templates in \pcpp\ we introduce a term a ``C++ concept''. This term is not to be confused with \rgp{concept} in 
projectional editing.

A C++ concept explicitly describes assumptions put on a template parameter by a template code. A template code is required to declare, which 
C++ concept a template parameter belongs to, before the parameter can be used. The C++ concept defines for the parameter the way the
parameter could be used in the template code through a mechanism, similar to giving a type to the parameter. This generalized type 
is described by the C++ concept itself. Thus the C++ concept describes, how the parameter can be used. In \pcpp\ the scoping 
constraints code is analyzing the C++ concept of the parameter, to determine, how it could be used.

\gr{templates}{Template Code Sample in \pcpp}{0.1}

The \fig{templates} demonstrates how a template code can be composed with the use of C++ concepts.

At first, the \cc{Comparable} C++ concept is declared. It puts a requirement on a type to have a function 
\cc{compare()} in its public section.
Next, a class called \cc{NumberWrapper} is declared to \cc{realize} the \cc{Comparable} concept. The
editor will check the class upon the requirement.

A template class \cc{OrderedList} declares a template parameter \cc{T} and specifies that it 
satisfies the \cc{Comparable} concept. This makes an object of type \cc{T} usable as if it was 
satisfying the requirements as declared by \cc{Comparable}. This is demonstrated in the \cc{OrderedList::compare()}
function.

When instantiating a template, it is checked if either the parameter given realizes the needed 
C++ concept, or, more flexibly, if it satisfies the C++ concept requirements, without declaring
the realization of the C++ concept explicitly.

The support for templates by \pcpp\ is, of-course, limited, when compared to traditional C++ facilities.
It is evaluated in the \rsec{templatesevaluation}.

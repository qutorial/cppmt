\subsection{Encapsulation and Inheritance}

% what to support in C++

Encapsulation and inheritance are considered here together, because from the language-engineering point of view, 
they just decide the access to class members. In other words, the \pcpp\ implementation has to track
encapsulation and inheritance related definitions and provide access to the class members accordingly.

% Problem

\subsubsection{Various Cases of Access Control}

In the \cpppl\ a number of ways to govern access control to class members exists. Before discussing 
the implementation of them in a projectional C++, I briefly review them with an example.

All members, a class has, are either declared in the class, or inherited from its base classes. 

The members can be accessed in a number of different locations in the code, which differ by the level of access they have
to the class members. Among these locations are the class methods, friend functions of the class, and
external to the class code. 

Each member can be declared with a certain visibility/access type, and the inheritance
of the base class can happen with one of the three inheritance modifiers. 

% The access depends on the object, on which the member is requested, and on its type as well.

\ms{inhdecl}{Declaration of two classes with a friend function}

The \fig{inhdecl} shows a declaration of two classes. The class \cc{A} has all three
public, protected and private fields. A function \cc{compare()} is declared to be a
friend function of  the class \cc{A}. 

The visibility plays no role for the friend function 
declarations themselves. That is why a decision was made to create a special 
section for friend declarations, called \cc{friends}. This section name is not generated anyhow in the 
resulting C++ text. This allows for all the friend functions to appear in one place, 
and be easily observable.

The class \cc{B} in the \fig{inhdecl} is inheriting publicly from the class \cc{A}, which means,
that public members of \cc{A} remain being public in \cc{B}. The class \cc{B} declares
a copy constructor.

Such declaration can be utilized as shown in the figure \fig{inhusage}.

\ms{inhusage}{Visibility resolution}

In the copy-constructor the visibility resolution happens after \cc{this} pointer
and after the \cc{original} object. Arrow expression and dot expression are used for
this. The first and second lines are making use of public and protected fields of 
the base class \cc{A}. The use of the private field is however not possible, since
private fields of a class are only accessible to methods of the same class and
friend functions. 

It is even not possible to input the not-allowed member, as the projectional editor
does not bind the text to anything, and it remains red. This means, no node in the \rg{ast}
is created and the code is in uncomplete erroneous state.


The \cc{compare()} function is declared in advance (\fig{inhdecl}) to be a friend 
function of the class \cc{A}. Thus, it is not a problem for this function to access
even private fields of \cc{A}, for comparison purposes in this example case.

The function \cc{printOut()} is not related anyhow to classes declared. Thus,
it represent ``external'' for the class \cc{B} code. Only public members
are accessible, but not protected or private. The attempt to input them, simply
fails, they are highlighted red, and are not bound to anything.

In this way the \pcpp\ gives for the programmer only a chance to input correct from the encapsulation
point of view constructions. As the members, accessible in each place of code, are provided by the 
\pcpp, instead of typing the member name, the programmer usually will have a choice from a short drop-down
list of options.

\subsubsection{Expressions to Address Class Members}

% MPS support

Members are usually accessed relatively to some object. The object can be designated as an expression of type class or a pointer to class,
in particular, \cc{this} expression. The resulting access represents nothing else, but an expression itself.

One of the greatest ideas in \jbmps\ to allow extensibility is \rg{concept} inheritance, \ref{modularity}, \ref{mpsconceptdeclaration}. 
Once a need to create a new concept arises, serving as a concept known before, the new concept has to inherit from the existing one, 
and this is almost all what has to be done. Thus inheriting the \mpsid{OoDotOrArrowExpression} \rg{concept} from the \mpslang{Expression} 
\rg{concept}, we get the ability to use the expression, designed for member access, wherever an expression in general can be used. 

The abstract \rg{concept} \mpsid{OoDotOrArrowExpression} serves as a parent for \mpsid{OoDotExpression} and \mpsid{OoArrowExpression}.
The commonality between the two, is
that an object is accessed in the left part, and a member is selected in the right part, as well as the way to decide the access to members.
The access is defined then, which left part is going to be possible in such expressions.


% Limitation
Within the class methods it is also possible in C++ to address class members as local variables. In the projectional implementation
described, it is not possible. Instead, \cc{this} expression has to be used. It makes typing a little bit slower, but allows to
easily distinguish between members of the class and other variables or functions.

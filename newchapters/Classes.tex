\declsec{C++ Object-Oriented Programming}{oop}

The \cpppl\ is a multi-paradigm programming language. The ability to support the object-oriented programming,
is incorporated via classes support. 

A class represents a new type in the \cpppl. Each class may have data in the 
form of fields, and behavior in the form of methods. Two types of methods are special for
C++ - constructors and destructors, they have special meaning and syntax.

Encapsulation is enabled via governing access permissions to fields and methods of a class.
The access control is governed with the creation of \cc{public}, \cc{protected} and \cc{private}
class sections.

Inheritance is implemented in C++ via allowing for each class to have one, or even many base classes.
Inheritance from a base class is performed under a certain access control modifier. There is no pure notion 
of an interface, but rather abstract classes are introduced.

Polymorphism is implemented via pointer-to-class type compatibility over inheritance-connected classes.

The implementation of these C++ features in a projectional editor environment is discussed in the
following sections.


\subsection{Class Declaration and Copying}
\label{section:classes}


% Enforcement of one section and only one.
% Enforcement of sections order
% Constructor markup
% Copy constructor and constructor generation


\subsubsection{Visibility Sections}
\label{visibilitysections}


Instead of declaring visibility type for individual class members (C\#, Java), visibility sections are created
in C++. 

The sections can be opened with a string \cc{private:}, \cc{protected:} or \cc{public:}
within a class declaration, and closed when another section is opened or when the class declaration
ends. This allows the user to open and close the same section multiple times and declare sections
without any particular order. This can be errorprone and confusive.

Various coding guidelines (\cite{cppclasslayout}, \cite{googlecppstyle}) exist to enforce some 
restrictions on the visibility sections.

In particular, the sections are allowed to be opened only once. This ensures, the
reader of the code will see the interface of the class (public section) in one place,
``contents'' of the class (private section) in one place, and opportunities to access members in
an inheriting class (protected section) in one place, without the need to search through
the whole class declaration.

Another typical requirement in coding standards, is the order of the sections. Usually the public 
section is required to be first, for the class users to see immediately the (public) interface,
the class provides.

\ms{webpage}{Sample class type declaration}


The \fig{webpage}, shows an example class declaration implemented in the projectional editor.
The \mpsid{Class} \rg{concept} has the visibility sections as children. Each section is given a separate role 
(see \ref{mpsconceptdeclaration}) and can appear 0 or 1 times. The editor for the \mpsid{Class} \rg{concept} 
orders the visibility sections so, that the public section always comes first, if present, 
followed by the private and protected sections.

The creation of a section is made with the use of \rgp{intention}. The user
uses \emph{Alt+Enter} combination on the class declaration to create visibility sections.
It should be more practical and fast for the user, compared to typing the keyword, colon and
indenting the result.

% Limitation
A question arises on how to support another way to represent a class, so that it will reflect
requirements from a different coding standard. And as a way to resolve it a definition of 
another editor for a class concept can be offered, Unfortunately, the current version of \jbmps\ 2.5
does not support a definition of multiple editors for the same concepts. This limitation however
is addressed in the newer 3.0 version, which was not tested completely with \mbdr\ on the time
this work was made.


\subsubsection{Constructors}
\label{classconstructors}

Constructors are special methods of a class, used to construct the class instances. 

Constructors have special syntax and no return type, being similar to class methods. Additional 
value, however, the constructors gain, when participating in type transformations. Namely, when
a constructor of a class \cc{B} exist from a type \cc{T}, instances of the type \cc{T} can be used whenever
the \cc{B} class instance is required. The constructor will be \emph{implicitly} called and a temporary object
of class \cc{B} is going to be created as a mediator. 

Thus constructors extend the type system of the C++ language, adding conversion rules to it. 
Since this type extension can not be easily observed, it is highly possible to get various 
\emph{run-time} errors or unexpected behavior. 

\cpp{Example of an Implicit Constructor Error}{implicit}

The \rl{implicit} demonstrates a simplified use case where the function \cc{print()} is invoked on \cc{int} without 
any compiler error, and the resulting behavior is unexpected.

To avoid similar situations, it is possible to deprecate participation 
of a constructor in type conversions, adding a word \cc{explicit} to the constructor
declaration.

The described problematic motivated the following decision. When a new constructor is created with one argument, 
it is by default declared to be explicit, the user must intentionally change it to get the type conversion 
behavior. Such behavior is safer by default. This is an example improvement to classical C++ which makes
the code understood better by novices and safer for them.

\subsubsection{Copying}
\label{classcopying}

In the \cpppl\ the programmer controls memory allocation fully on his/her own. This affects the way of copying fot the class
instances.
In C++ a programmer should define two methods for a class: a copy constructor and an assignment operator.
These two methods work when assignment like \cc{a = b} happens, when instances of a class are passed by value to a function, 
when one object is initialized with a value of another object and so on.

C++ serves here sometimes dangerously generating default copy constructor and assignment operator, which by default represent
a bitwise cloning of an object. This can lead to problems. For example, if a pointer value is getting copied bitwise in a second
instance and is deallocated twice in destructors.

\cpp{Need in a Custom Copy constructor}{copydeath}

The \rl{copydeath} demonstrates a program which crashes upon execution as destructors of \cc{a} and \cc{b} are deallocating 
memory with the same address, after default copy-constructor copies the address from \cc{a} into \cc{b}.

To avoid the described problem, the programmer has to either define a proper copy-constructor or forbid copying of the objects for
the class. The same applies to the assignment operator. Many standards require the two functions to be implemented in sync, i.e. 
implementing the same semantic behavior, \cite{ooocpp}. This can be performed in an elegant way by implementing the assignment 
operator first and reusing it in  the copy-constructor.

When not providing the copying behavior it comes logical to disable also the assignment behavior. 
This is done by a trick of declaring the corresponding functions in the private ares, without implementing them
(as they never get called). To visually explain such design and make it handier to implement, specialized known macros exist in 
various libraries, for example \cc{DISALLOW\_COPY\_AND\_ASSIGN} or \cc{Q\_DISABLE\_COPY}, \cite{googlecppstyle}, \cite{qobjref}.
An alternative approach is the use of \cc{boost::noncopyable} from the 
boost library, \cite{boost}.

The use of macros in C++ appears often in similar cases, in order to perform some language-engineering tasks to add the missing
features to the language (copying or assignment deprecation in this case). 
Macros bear pure textual nature, and are processed by the pre-processor. Some negative effects may 
come out: need to preprocess reduces the speed of compilation and hides the resulting code from a programmer. Macros lead
to error prone programming, as no type checks are possible. And macros make code less analyzable by automatic analyzers.

\ms{copyable}{Hinting about Copyable and Assignable Class Properties}


The projectional editor allows for another solution, different from macros, or the use of a new library.  
In order to provide some support for the  programmer regarding the copying issue, the \pcpp\ hints on the class declaration its 
assignable and copyable properties, \fig{copyable}, and generates by default the declarations of copy-constructor and assignment operator 
declarations, when the class is created.

The copy constructor and the assignment operator are recognized by the \pcpp. Two \rgp{intention} are provided on the 
\mpsid{Class} \rg{concept} to forbid or allow copying. The forbidding \rg{intention} imitates the macros mentioned 
above, but displaying and explaining the implementation to the user\fig{notcopyable}. The implementation of the intention 
consists of moving the declarations of two functions to the private section of the class. The allowing \rg{intention} moves
the declaration back to the public section, or creates them. The check for implementation provided for a method flags
the two declarations appropriately, \ref{implementedcheck}, to finish the process.

\ms{notcopyable}{Class Made not Copyable by the \emph{Forbid Copying} Intention}

The \jbmps\ supports the \pcpp\ implementation by providing read-only model accesses, special parts of the editor concept,
 \ref{mpseditor}, by which the hinting is implemented. The \rgp{intention} allowed manipulations on the \mpsid{Class} \rg{concept}, 
which made it possible to  automate the allowing or forbidding of copying/assignment, making the implementation clear 
to the user, without the use of macros or libraries. 

The whole work the programmer needs to perform to forbid copying and assignment contains of a call to an \rg{intention}, 
one key-stroke. There is no need to include a header file with macros, and look up the documentation for them, using 
them in a right way afterward (symmetric macro requires exactly one parameter - the class name, and it has to go in 
the private class section). The boost library, \cite{boost}, providing functionality to disable copying is often considered to be to
heavy-weight to include, when it goes about little tasks, like the one described in this section.

\subsection{Encapsulation and Inheritance}

% what to support in C++

Encapsulation and inheritance are considered here together, because from the language-engineering point of view, 
they just decide the access to class members. In other words, the \pcpp\ implementation has to track
encapsulation and inheritance related definitions and provide access to the class members accordingly.

% Problem

\subsubsection{Various Cases of Access Control}

In the \cpppl\ a number of ways to govern access control to class members exists. Before discussing 
the implementation of them in a projectional C++, I briefly review them with an example.

All members, a class has, are either declared in the class, or inherited from its base classes. 

The members can be accessed in a number of different locations in the code, which differ by the level of access they have
to the class members. Among these locations are the class methods, friend functions of the class, and
external to the class code. 

Each member can be declared with a certain visibility/access type, and the inheritance
of the base class can happen with one of the three inheritance modifiers. 

% The access depends on the object, on which the member is requested, and on its type as well.

\ms{inhdecl}{Declaration of two classes with a friend function}

The \fig{inhdecl} shows a declaration of two classes. The class \cc{A} has all three
public, protected and private fields. A function \cc{compare()} is declared to be a
friend function of  the class \cc{A}. 

The visibility plays no role for the friend function 
declarations themselves. That is why a decision was made to create a special 
section for friend declarations, called \cc{friends}. This section name is not generated anyhow in the 
resulting C++ text. This allows for all the friend functions to appear in one place, 
and be easily observable.

The class \cc{B} in the \fig{inhdecl} is inheriting publicly from the class \cc{A}, which means,
that public members of \cc{A} remain being public in \cc{B}. The class \cc{B} declares
a copy constructor.

Such declaration can be utilized as shown in the figure \fig{inhusage}.

\ms{inhusage}{Visibility resolution}

In the copy-constructor the visibility resolution happens after \cc{this} pointer
and after the \cc{original} object. Arrow expression and dot expression are used for
this. The first and second lines are making use of public and protected fields of 
the base class \cc{A}. The use of the private field is however not possible, since
private fields of a class are only accessible to methods of the same class and
friend functions. 

It is even not possible to input the not-allowed member, as the projectional editor
does not bind the text to anything, and it remains red. This means, no node in the \rg{ast}
is created and the code is in uncomplete erroneous state.


The \cc{compare()} function is declared in advance (\fig{inhdecl}) to be a friend 
function of the class \cc{A}. Thus, it is not a problem for this function to access
even private fields of \cc{A}, for comparison purposes in this example case.

The function \cc{printOut()} is not related anyhow to classes declared. Thus,
it represent ``external'' for the class \cc{B} code. Only public members
are accessible, but not protected or private. The attempt to input them, simply
fails, they are highlighted red, and are not bound to anything.

In this way the \pcpp\ gives for the programmer only a chance to input correct from the encapsulation
point of view constructions. As the members, accessible in each place of code, are provided by the 
\pcpp, instead of typing the member name, the programmer usually will have a choice from a short drop-down
list of options.

\subsubsection{Expressions to Address Class Members}

% MPS support

Members are usually accessed relatively to some object. The object can be designated as an expression of type class or a pointer to class,
in particular, \cc{this} expression. The resulting access represents nothing else, but an expression itself.

One of the greatest ideas in \jbmps\ to allow extensibility is \rg{concept} inheritance, \ref{modularity}, \ref{mpsconceptdeclaration}. 
Once a need to create a new concept arises, serving as a concept known before, the new concept has to inherit from the existing one, 
and this is almost all what has to be done. Thus inheriting the \mpsid{OoDotOrArrowExpression} \rg{concept} from the \mpslang{Expression} 
\rg{concept}, we get the ability to use the expression, designed for member access, wherever an expression in general can be used. 

The abstract \rg{concept} \mpsid{OoDotOrArrowExpression} serves as a parent for \mpsid{OoDotExpression} and \mpsid{OoArrowExpression}.
The commonality between the two, is
that an object is accessed in the left part, and a member is selected in the right part, as well as the way to decide the access to members.
The access is defined then, which left part is going to be possible in such expressions.


% Limitation
Within the class methods it is also possible in C++ to address class members as local variables. In the projectional implementation
described, it is not possible. Instead, \cc{this} expression has to be used. It makes typing a little bit slower, but allows to
easily distinguish between members of the class and other variables or functions.



\subsection{Polymorphism}

There are several ways to achieve polymorphic behavior in C++. The purists of the language differentiate 
between the polymorphism based on virtual functions or based on templates. More general opinion can include
in the notion of the polymorphism also functions overloading and operator overloading, also called 
\emph{ad hoc} polymorphism, polymorphism for concrete one purpose. 
Occasionally various operations with \cc{void*} type are also classified as a polymorphic programming.

In this section we are writing only about the class-related virtual functions polymorphism, and the way it is 
implemented in \pcpp.

\subsubsection{Virtual Functions Polymorphism in C++}
\label{cpppolydefs}

At first, we describe the polymorphism, as it is implemented in the \cpppl, pointing out the places, where it could be improved.

Dislike many other popular object-oriented programming languages, e.g. in Java, in the C++ there is no pure notion of
an interface. Instead, a base class, its public part, is used as an interface declaration for the descendants. 
Functions designated to be a part of the interface must be declared \cc{virtual} and they can be overloaded in
the subclasses.

The virtual functions in the base-interface class can be implemented as well, providing some ``default''
common enough behavior. Otherwise they are left \rg{purevirtual}, meaning that no implementation is provided
and the pointer to the function in the table of pointers to virtual functions is zeroed. The syntax
for \rg{purevirtual} functions is rather not obvious\footnote{especially when not knowing about the zero pointer value semantics}
and a bit cumbersome requiring to type one reserved word, one punctuation sign and one digit to 
express a simple fact of pure virtuality or, simply, absence of implementation, \rl{purevirtual}.

\cpp{Pure Virtual Function Syntax Example}{purevirtual}

The approach of classes with virtual functions as interfaces is more flexible compared to languages with the notion 
of an interface is directly introduced.
In C++ it is possible to create partially implemented base classes, what can not be done when implementing
and interface in Java or C\#, where the approach is ``all or none'' regarding the implementation of interface
functions. The presence of interfaces in the language can be though considered a more clean way to program.

In order to implement a declared in a base class function a descendant must declare and implement
a virtual function, matching the full (including a return type) signature of a declared in the base class function, \rl{override}. 
The connection to the ``interface'' function declaration stays subtle however. It is not immediately clear, whether the 
declared new function in the descendant is an override of an existing in the base class function or an independent declaration
of an entirely new function in the descendant class. 

This knowledge affects the changing process greatly, as the override should change from the interface, 
together with all the implementations. The absence of a clear, explicit override syntax we call here an ``\rg{ovsa}''.

Whenever a class and all of its ancestors do not provide an implementation of a certain virtual function,
created as a pure virtual in the declaring class, the class is called an \rg{abstract} class. It is not possible
to construct instances of an \rg{abstract} class. C++ however does not have any special syntax to explicitly declare a 
class \rg{abstract},  \rl{purevirtual}.
The programmer usually has to be aware (from documentation, implementation, or, the worst case, compilation
errors) whether a given class is \rg{abstract}. We will call this phenomenon an ``\rg{acsa}''.

Overriding a function is an active action of the programmer, and it is initiated by the 
programmer. I.e. the programmer want to state, that the new function is designed to override an exiting one, and 
which is the overridden function. The abstract property of a class, oppositely, is not a quality a programmer directly
gives to a class. It can be rather deduced from the analysis on the base classes automatically, by editor. So in this case
no actions are needed from the programmer side. Because of this two similar absence of syntax phenomena are resolved or
have to be improved on differently from the \pcpp\ point of view.

In order to use an interface, declared in some class, the using code has to get a pointer to an instance
of any inheriting the interface descendant class instance. Thus type system has to allow a pointer
to the descendant to be treated in the way as a pointer to the base class would be. The same should hold, 
normally, for the reference types, but it is not used very often on practice, and is omitted in the implementation.

As this typing rule represents the core of polymorphic behavior, we will start from it below, describing further the
polymorphism in the \pcpp\ implementation.

\subsubsection{Special Typing for a Pointer to Class}
\label{pointertoclasstyping}

The problem solved here is enabling the usage of pointers to descendants instead of pointers to ancestors,
\rl{polycall}.

\cpp{Example Usage of a Pointer to Descendant Class}{polycall}

The \mpsid{PointerType} \rg{concept} is implemented in the \mpslang{pointers} language in \mbdr. 
Following the goal of non-invasive changes to \mbeddr\ the \pcpp\ implementation needs to add the 
typing rules for pointers \emph{to classes} without changing the  \mpslang{pointers}  \mbdr\ language, where the 
typing system for pointer type is defined.

In the case when a type of a pointer to a base class is expected, a pointer to a subclass should also be accepted, like
in the \rl{polycall}.

The type system of the pointer language will try to check the compatibility of the two pointer types. It will
fail to do so, as the \mbdr\ languages are not aware of the \pcpp\ extensions\footnote{\mpsid{Class} \rg{concept} in this case} 
by design.


In a case like this a replacement rule can be used, see \ref{mpsts}. The \pcpp\ provides such a replacement rule for the 
pointer type, which checks, whether a class pointed to, is a passing descendant of the class required by an expression, 
where the pointer was used, and performs the necessary typing.

In the \rsec{extensibility} we discuss more on the approach taken here, its limitations and 
the potential ways to overcome them.

\subsubsection{Overriding a Virtual Function}
\label{overridefunction}

The \rl{override} demonstrates, how overriding of a virtual function happens in C++.
The function \cc{getArea()} is initially defined in the class \cc{Shape} and is then overridden
in the class \cc{Circle}. The \rl{override} demonstrates as well the \rg{ovsa} in C++.

\cpp{Example of an Overridden Function - Text Code}{override}

In the \fig{override} the same example is demonstrated, but in the \pcpp\ implementation.
In fact, the code shown in the \rl{override} can be generated 
from the demonstrated projectional sample in the \fig{override}, by invoking all \Rgp{textgen} in \jbmps.

When a method declaration is created, it can be set to be an override of a method in a base class.
Right after an empty method declaration is created, it can be set to be an override, so that the only
thing which stays, is to pick a method from a base class to be overridden. The override is automatically
named, the parameters and the return type are set accordingly, the \cc{virtual} property is immediately
set. This is yet another work saver for the programmer.

After the override has been linked to the overridden method, the projectional editor checks, if 
the override full signature stays precisely the same as the one of the overridden method. Thus if the 
latter changes, it is going to be indicated in the former.
The overridden method is shown next to the override declaration \fig{override}, which 
compensates on the \rg{ovsa}.

\ms{override}{Example of an Overridden Function - Projection}

The additional, out of C++, visual syntax in the \pcpp, should not confuse the reader. It is only 
present in the projectional editor, and, when an \rg{ast} is generated into a text code, the regular
C++ syntax is achieved. However in this case the \rg{ast} stores \emph{more} than needed to 
generate and compile the C++ code. 

This subsection is one of the examples, that storing more information can be useful,
it is generalized in the \rsec{genprinciples}.


\subsubsection{Pure Virtual Functions}
\label{purevirtualfuncs}

In the example above, \rl{override}, one improvement to the code can be made. As the \cc{getArea()} for a random shape can not 
be determined, it makes sense to make the \cc{getArea()} function \rg{purevirtual}. As said before, a \rg{purevirtual} 
function has no implementation in the declaring class, and serves only the overriding purpose. Semantically in C++ it sets the 
pointer in \cc{vptr} to 0 and thus has reflecting it syntax, see the similar example in the \rl{purevirtual}.

This syntax can be seen as not obvious, as it reflects more the under-the-hood implementation of the mechanism,
rather then the original programmer intention to built a basis for an overrides chain, while taking advantage of 
polymorphism.

Additionally, as discussed above, declaring a \rg{purevirtual} function requires a significant amount of the syntactical
overhead. 

These were the disadvantages of the \cpppl\ the \pcpp\ tries to improve on.

In the \pcpp implementation, one \rg{intention} is reserved to make a function \rg{purevirtual}. The \rg{intention} automatically
sets the needed pure virtual and virtual flags of the function declaration, and the projection changes. That is why out of the 
statement \cc{virtual double getArea() = 0;} the programmer has to input only the name \cc{getArea}, pick the type \cc{double}, and
toggle the pure virtual intention for the declaration in the \cc{Shape} class.

The result of the intention work in projection can be seen in the \fig{purevirtual}.

\ms{purevirtual}{Example of a Pure Virtual Method - Projection}

The word \cc{pure} is added by the projectional editor. This makes the reading of the code easy and natural.
The \cc{ = 0} part is preserved for the C++ programmer with habits, used to the original C++ syntax. And, as in many other cases, 
the semicolon is omitted as it is nothing but syntactic help to the compiler, which does not have to appear 
in the projection. 

This example demostrates, how the lower level syntax can be made more readable. The general principle on it is formulated in
the \rsec{genprinciples}.

\subsubsection{Abstract Classes}
\label{abstractclasses}

If any of the \rg{purevirtual} functions, in the inheritance chain leading to a class, is not implemented by this chain, 
or inside the class itself, the class is called an \rg{abstract} class. It is not possible to construct an instance of an
\rg{abstract} class, as such classes have not implemented methods.

The programmer has to know which classes are \rg{abstract},
and are intended for inheritance and further implementation only. One of the reason for this, is to not to try instantiation of
the abstract class. Another reason is to exactly identify \rg{abstract} classes, designed to serve as extension points.

Above we discuss the \rg{acsa} in C++. It leads to the need for the programmer to determine somehow him/herself 
if a given class is \rg{abstract}, the source code representation of a class does not give any information on it, 
unless some naming conventions require \rg{abstract} classes to be named specially.

An improvement to this situation would be a behavior, when an editor can perform an analysis and 
determine if a class is \rg{abstract}, hinting on it to the user. In the case of the projectional
editor, this analysis is especially computationally efficient\footnote{in comparison to textual editors, with the need to 
parse, see the \rsec{comparison} on comparison to the textual approach, and \rsec{analysesandcomplexity} on complexity of analyses}, 
as the \rg{ast} is readily available, and the quick  analysis can be performed by a simple recursive algorithm on the inheritance chains.
%TODO Make an appendix with the algorithm and reference it

After an \rg{abstract} class has been determined, it is possible to modify the editor representation for it, 
and show that it is abstract, \fig{abstract}.

In this example a typical class hierarchy is created to support user interface programming. A user of this \rg{api}, when 
searching for a button, could try using the \cc{Button} class, which is designed to be \rg{abstract}, and serve as a 
base for the further implementations, e.g. \cc{PushButton} in the example, or check boxes, radio buttons and similar,
later.

\ms{abstract}{Determining Abstract Classes}

The projectional editor, however, checks on the fly, if a certain class is abstract, adding a special \cc{abstract} word 
in front of its declaration. This makes the reading easier, and allows for quicker understanding of the code. 

This example demonstrates a general principle, see  the \rsec{genprinciples}, of a advised practice to 
perform quick analyses and inform the user.

Additionally, the creation and usage of abstract class instances is checked, and forbidden by type analysis. This is 
described separately in the \rsec{advanced}.


% To be used later when writing about type conversion.

% Constructors among the special meaning of creating 
% instances of the type class, are also participating in the type conversion mechanisms.
% Namely, if there is a constructor a class of \cc{B} from a class \cc{A}: \cc{B(A obj)} or 
% similar, any object of the class \cc{A} can participate in places, where an object of the 
% class \cc{B} is requested originally, as implicit type conversion takes place, invoking the 
% \cc{B(A obj)} constructor.


\declsec{Jetbains MPS}{mps}

\jbmps\ stands for Jetbrains Meta Programming System. In this \Rg{ide}-like software it is possible to develop \Rgp{dsl}.
The approach taken in the \jbmps\ is rather unique, and it is considered to be advantageous in many ways, \cite{Voelter:MoDELS:2010}.

% No way to extend mbeddr otherwise
% Thus \jbmps\ is a good suitable technology for the practical implementation of the \MT\ goal.

In general, the way the language is describe in \jbmps\ coresponds to the way, described in the \rsec{modularlang}.
Here I will describe the process in practical details, as it is crucial for understanding the practical part of this \MT.

In this section I will go through a definition of one \rg{concept}, describing the facilities, \jbmps\ provides to support
it. Each \rg{concept} is described by several views on it. 

\msnozoom{mpsif}{\jbmps\ User Interface, \cc{if} Statement \Rg{concept}}

Among such views there are \rg{concept} declaration (or language structure) view, editor view, behavior view, constraints view, type system view,
generator view and few more views. As a language consists mostly of the included in it concepts, 
the whole language is presented by the mentioned views as well, where each view on the language contains all views of the kind on each language \rg{concept}. 
Left part of the \fig{mpsif} demonstates the views on the \mpslang{statements} language.

The representation of a new language created in concepts and views to them, present a seamless approach to creating 
a new language with a projectional editor for it.

The \mbp\ is a software, separate from \jbmps\, but based on it, representing an extensible C language implementation with extensions.
I will use \rgp{concept} from the \mbp' throughout this section as examples to demonstrate MPS. Different C language parts are going to be 
decomposed into \rgp{concept} and these concepts are going to be defined using \jbmps. The reader should not confuse though the \mbp\ and 
\jbmps\ itself: the former is a software, developed in the latter and is used to demonstrate the latter.

\subsection{Concept Declaration}

\Rgp{concept}, as it is described in the \rsec{modularlang} represent a class-like types for nodes of an \rg{ast}. This terminology is kept in \jbmps\ and 
MPS concept has the same meaning as \rg{concept} term used in the \rsec{modularlang}. I use the term ``concept'' both in general, to describe an \rg{ast} node type,
and in particular referring to an MPS concept.

% Repeats \rsec{modularlang}
%
% One defines a language in it not through the canonical grammar approach, but instead through defining so called concepts, and relationships 
% between them similar to those in ER diagrams. A logical part of a language can be made a concept. 
% Related to C++, class, method declaration, field declaration can be represented as a concept. 

The \fig{mpsif} on the right part demonstrates a declaration of \mpsid{IfStatement} concept from the \mbdr\ \mpslang{statements} language.
It corresponds to the \cc{if} statement of the C language.

At first, the \rg{concept} is named, and a \rg{baseconcept} is defined. The \mpsid{IfStatment} \rg{concept} inherits from the \mpsid{Statement} \rg{concept}.
This allows the \mpsid{IfStatement} to be used at any place at which the \mpsid{Statement} could be used, and inherits like in object oriented programming
all data and behavior of the \mpsid{Statement}.

Next, it is defined, which interfaces the \rg{concept} is going to implement. For example, by implementing \mpsid{ISteppableContext}, the \mpsid{IfStatement}
supports the \mbdr\ debugger when stepping in the body of the \mpsid{IfStatement}.

The ``instance can be root'' property defines, if it is meaningful, to create a \rg{concept} without a parent \rg{concept} for it. In the case
of the \mpsid{IfStatment} it does not make sense, as the statement should belong to some block. The \cc{true} value can be used, e.g. for modules,
as they do not have any outer concepts, and can be seen as a document in \jbmps.

The ``properties'' part defines if the described \rg{concept} instances should have some primitive type data fields (string, boolean, int).
An example of a property could be a \mpsid{name} property of a variable declaration. The \mpsid{IfStatement} \rg{concept} does not specify any 
properties, neither does it inherit any from the \mpsid{Statement} concept.

The ``children'' section describe which nodes can be children on the \rg{ast} of the \mpsid{IfStatement}. Each child is assigned with a role and cardinality. 
For example, the \mpsid{IfStatement} should have exactly one child of \rg{concept} \mpsid{Expression}, it has a role ``condition''.

The ``references'' section describes in the similar way as in the ``children'' section, which nodes could be referenced by the node of a given \rg{concept}.
Referencing can be used, to bind a given node, to a node, located somewhere else on the \rg{ast}. As an example, a variable usage in expression shall reference
the variable declaration, to express precisely, which variable is used.

Finally, some attributes of a \rg{concept} follow, which do not have a primary importance for the discussion here. The ``alias'' is used to 
name a \rg{concept} in a short way, to allow for quick instantiation in the editor. The ``short description'' is shown to hint a programmer on the
alias meaning. A \rg{concept} can be made abstract in the ``concept properties'' section. Abstract \rgp{concept} are purely used in inheritance 
to create other non-abstract concepts with a common parent.

\jbmps\ separately allows to define so-called \rgp{interfaceconcept}. Interface concepts are \rgp{concept} which can not be instantiated, but 
which serve as a base for inheritance and implementing and interfaces as Java classes do. A \rg{concept} can have only one base concept, but can
inherit from/implement many interface concepts.

\subsection{Editor View}

The editor view, \fig{mpsifeditor}, is designed to give a look for a node of a \rg{concept}, and a way to input it. This is where the projection of an \rg{ast} node is defined. 
As editors are mostly defined to look like text, a program in the \cpppl implemented in MPS looks almost like a regular C++ code.

In the editor view one defines an editor for a \rg{concept}. \jbmps\ introduces here a bit confusing terminology, calling one editor for a given
\rg{concept} concept an ``editor concept''. Thus an error message ``An editor concept not found for a concept X'' would mean that no editor
has been defined for the \rg{concept} \mpsid{X} in the editor view for it. In this work I call the content in the editor view for a \rg{concept} \mpsid{X} an 
``editor for the \rg{concept} \mpsid{X}''. The same terminology applies to all the following views related to a single \rg{concept}.

\msnozoom{mpsifeditor}{Editor View for the \mpsid{IfStatement} \Rg{concept}}

The editor for a \rg{concept} defines the visual representation for a node of the concept, using special syntax. For example, \fig{mpsifeditor} defines
and editor for the \mpsid{IfStatement}. At first the ``constant'' non-changeable by a programmer text is given, which is ``if'' and ``(''. Then
the child \mpsid{condition} is referenced, so that after the ``if('' the user will be able to input an expression for the condition of the \cc{if} statement.
The editor can be configured to show or hide some parts of a node, depending on some condition, e.g. hiding \cc{else} part of the \cc{if} statement, if it 
is not defined anyhow. Editor are also responsible for all the interaction, a user experiences when editing a code in \jbmps.

\subsection{Behavior View}
The behavior view, can be used to define certain behavioral methods for a concept. A concept is represented there similar to a Java class, and
it is possible to define the methods in a Java-like language. \Rg{concept} inheritance is taken into account like in Java. 

A \rg{concept} constructor can be defined there to initialize by default a newly created node of a \rg{concept}.

\msnozoom{mpslocvarrefbeh}{Behavior View for the \mpsid{LocalVarRef} \Rg{concept}}

The \fig{mpslocvarrefbeh} shows the behavior of the \mpsid{LocalVarRef} \rg{concept}. This \rg{concept} represents in the
\mbp\ C language an expression, referencing a local variable. The local variable declaration is stored as a reference \mpsid{var}
in the \rg{concept} nodes.

Two convenience methods are defined for the \mpsid{LocalVarRef} \rg{concept}, to get an easy access to the local variable properties.


\subsection{Constraints View}
The constraints view can be used, to limit in a desirable way \rg{concept} property values and relationships, a node of the \rg{concept} can 
have to other nodes.

\msnozoom{mpsidnameconstr}{Editor View for the \mpsid{IfStatement} \Rg{concept}}

The \fig{mpsidnameconstr} shows constraints defined for a property \mpsid{name} of the \mpsid{IIdentifierNamedConcept} \rg{interfaceconcept}.
All \rgp{concept} which have a name, which must confirm to the identifier naming restrictions, can implement the \mpsid{IIdentifierNamedConcept},
to immediately get the desired characteristic.

The \mpsid{name} property is programmatically restricted by the use of Java-like code snippet, \fig{mpsidnameconstr}. In a similar way
relationships to children and referents can be restricted.

It is possible to create a pure Java class withing an \jbmps\ language, and use it in almost any \rg{concept} view in the \jbmps. 
In the \mpsid{IIdentifierNamedConcept} \rg{interfaceconcept} constraints the \cc{CIdentifierHelper} class was used to check the
name property on collision with the C language keywords.

\subsection{Type System View}
\jbmps\ has a special support for creation of typed languages. Types are mainly used in expressions. An expressions may count on 
a certain sub-expression to have a given type, or a type, compatible with it. Whenever the expectation does not meet the reality,
a warning or error can be displayed to a programmer. 

\msnozoom{mpsternaryts}{Type System View for the \mpsid{TernaryExpression} \Rg{concept}}

A \rg{concept} representing nodes which can have type in \mbdr\ inherits from \mpsid{ITyped} interface concept. In the type system view
certain rules must be defined with the use of a special language, which define the type or type comparison rules for a \rg{concept}.
Moreover, the type system language can be used to infer a type for a given node, \fig{mpsternaryts}.

The \fig{mpsternaryts}, demonstrates the use of the \jbmps\ type system language to infer a type of the \mpsid{TernaryExpression} \Rg{concept}
node. The syntax is on the one hand self-explanatory when reading, but on the other hand could be rather confusing when crafting.

\subsection{TextGen View}

To make use of the code in the projectional editor, further tools must be invoked on it, e.g. parser, compiler, etc.
Normally they work with a textual representation of the same code. In order to obtain the textual code from the \rg{ast} in the projectional
editor generators are invoked. 

Generators can be of two kinds. The kind of generator is dedicated to transform an \rg{ast} in one \rg{dsl} into
another \rg{ast} represented in a, usually, lower-level language. The second kind of generators is dedicated to transform an \rg{ast} into text.
Such generators are called ``\rgp{textgen}'' in \jbmps.

\msnozoom{mpsbinexptextgen}{TextGen View for the \mpsid{BinaryExpression} \Rg{concept}}

The \fig{mpsbinexptextgen} demonstrates how a node of a \mpsid{BinaryExpression} \Rg{concept} is converted to text. It is noteworthy 
to say, that when rendering to text the \mpsid{left} and \mpsid{right} sub-expressions, the corresponding \rgp{textgen} are invoked,
making the text generation recursive.


\subsection{Generator View}
The generator view is one of the most complex \jbmps\ views. It servers the transformation purposes of one \jbmps\ language into another one.
This view does not have a strong connection to the present work, as the generation of the projectional \cpppl\ is mostly performed by \rgp{textgen}.

\subsection{Intentions}
\jbmps\ \rgp{intention} are special procedures which can be used for automatic manipulations on the \rg{ast} with a node of a given \rg{concept}.

\msnozoom{mpsconstintention}{Toggle Const Property for a \mpsid{Type} \Rg{intention}}

The \fig{mpsconstintention} demonstrates an intention, used to modify the \cc{const} property of a given type. 

For an \rg{intention} to be defined, one has to name it, specify a \rg{concept}, to which the \rg{intention} is applicable,
provide a textual description for it, and finally specify the desired effect.

Intentions are accessible in the projectional editor from a context menu, when focused on a target node. They represent 
a useful mechanism to support a programmer with various automations, including automatic code generation.

\subsection{Other MPS Instruments}

% TODO 
% Refactorings
% Actions

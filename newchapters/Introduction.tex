\declchap{Introduction}{intro}

% Context

In embedded programming the C++ programming language is widely spread, \cite{embedlangs}. Being a general purpose 
programming language, C++ does not provide any special support for an embedded systems programmer. 

By changing  the language itself, together with a tool set for it, it is possible to get a better environment 
for specifically embedded programming. As an example, a subset of C++, called \Gls{emcpp} can be brought, \cite{emcpp}.
The approach taken in Embedded C++, namely, stripping very many core features of C++ off, allows for a higher degree of
optimizations by compiler possible. Simpler C++ was intended to (\cite{stripepp}) allow higher software quality through
better understanding of the limited C++ by programmers, higher quality of compilers, better suitability for the embedded
domain. This approach, however, has been criticized by the C++ community (\cite{stremcpp}), specifically for the inability 
of the limited language to take advantage of the C++ standard library, which requires the C++ language features, absent in 
Embedded C++.

Another approach to modify a language, to get it more suitable for the embedded development, is the opposite to the one,
taken in Embedded C++. It consists of extending the language with constructions specific to the domain. Such approach is
taken, for example, in the \gls{mbeddr} (\cite{2012_voelter_mbeddr_extensible_c_based_language_and_ide_for_embedded}),
for C programming language. As an example of extensions to C language state machines and decision tables can be brought.

Additionally, the integrated development environment (\gls{ide}) can be improved to support domain specific development. 
Various analyses can be built in into the code editor in order to detect inconsistencies, or simply ``dangerous'' constructs.
Certain code formatting, or standard requirements could be enforced, and many more.

% Problem 
A mixture of all three techniques could be combined in an attempt to achieve a ``better'' C++ for embedded development.
A special \gls{ide} can be created together with a new C++ language flavor, which prohibits the most ``dangerous'' C++
constructs, and allows for modular creation of extensions.

Before the chance to create extensions to the C++ language, the C++ language limited flavor itself together with a special \gls{ide}
have to be created. This is the problem to be solved in this Master Thesis.

% Approach

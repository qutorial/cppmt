\declchap{Introduction}{intro}

% Context

In embedded programming the C++ programming language is widely spread, \cite{embedlangs}. Being a general purpose 
programming language, C++ does not provide, however, any special support for an embedded systems programmer. 

By changing  the language itself, together with a tool set for it, it is possible to get a better environment 
for a dedicated domain, for example, specifically for embedded programming. 

The first possible approach is dropping some language features, to get the language, which is simpler. 
As an example, a subset of C++, called \Rg{emcpp} can be brought, \cite{emcpp}. The approach taken in Embedded C++ is 
omitting very many core features of C++ off, allows for a higher degree of optimizations by compiler possible. 
Embedded C++ was intended to allow higher software quality through better understanding of the limited 
C++ by programmers, higher quality of compilers, better suitability for the embedded domain, \cite{stripepp}. 
This approach, however, has been criticized by the C++ community, specifically for the inability of the 
limited language to take advantage of the C++ standard library, which requires the C++ language features, absent in 
Embedded C++, \cite{stremcpp}.

The second approach to modify a language to get it more suitable for the embedded development, consists of extending 
the language with constructions specific to the domain. Such approach is taken, for example, in the \rg{mbeddr}, to improve on
the C programming language, \cite{2012_voelter_mbeddr_extensible_c_based_language_and_ide_for_embedded}. 
Extensions to C language developed in \rg{mbeddr} include state machines and decision tables.

A special language engineering environment is used to support modular and incremental language development in \rg{mbeddr}, 
\Rg{mps}. The language under development is split in a special class-like items, called \glspl{concept}. As an example of a concept expression 
can be taken. Over the inheritance mechanisms, it is possible to extend languages, providing new concepts as children of the existing ones.
For example, expression concept can be extended to support new sort of expressions, e.g. decision tables.

Building a general purpose programming language in a language engineering environment brings a basis to 
develop domain specific extensions to a well-known general purpose language.

Additionally to the language modification, the \rg{ide} can 
be improved to support domain specific development. Various analyses can be built in into the code 
editor in order to detect inconsistencies, or, simply, ``dangerous'' constructs, and inform the programmer.
Certain code formatting, or standard requirements could be enforced as well, and many more.

% Problem 
A mixture of the two approaches could be used in an attempt to achieve a ``better'' C++ for embedded development.
A special \rg{ide} can be created together with a new C++ language flavor, which supports the embedded C++ programmer.
This is the problem to be solved by this Master Thesis.

The new language together with the new \rg{ide} can later serve as a basis for extending the C++ programming 
language with domain specific constructs for embedded programming. Creation of these extensions lies out of scope
for this Master Thesis, and is left for further research.

% Approach
The approach taken in this work goes further into exploring the language modularity on the basis of \jbmps\. While building the \cpppl\ itself 
with the goal of embedded domain specific extensions in mind, the C++ itself is being built itself as an extension to the C programming language,
provided by the \rg{mbeddr}.

Although C++ is a separate from C language, the high degree of similarity allows to make use of the C programming language,
implemented by the \rg{mbeddr} as a foundation. Not only reuse of the basic C is achieved, but also the embedded extensions from
the \rg{mbeddr} are immediately supported by the newly built C++.

This work explores further the support, provided by \jbmps\ for the modular language construction, \cite{2012_ratiu_modular_dsls_and_analyses},
and reviews it from the architectural point of view.

The \cpppl\ is provided with a number of automations and analyses for it. The automations include code generation and structuring.
They are implemented as a programming on the \rg{ast} in a Java-like programming language. 
The analyses and automations intend to improve quality, security and better understanding of the code.

In the \jbmps\ \rg{api} it is not explicitly defined, when the analyses and checks take place, how much of 
the computational resource they can take advantage of. This may affect the overall \rg{ide} performance, 
as the analyses complexity may be high. The question of analyses run-time and complexity is raised and discussed in this work.
\declchap{Introduction}{intro}

% Context

In embedded programming the C++ programming language is widely spread, \cite{embedlangs}. Being a general purpose 
programming language, C++ does not provide any special support for an embedded systems programmer. 

By changing  the language itself, together with a tool set for it, it is possible to get a better environment 
for specifically embedded programming. 

The first possible approach is stripping some language features of, to get the language, which is simpler. 
As an example, a subset of C++, called \Gls{emcpp} can be brought, \cite{emcpp}. The approach taken in Embedded C++ is 
stripping very many core features of C++ off, allows for a higher degree of optimizations by compiler possible. 
Embedded C++ was intended to (\cite{stripepp}) allow higher software quality through better understanding of the limited 
C++ by programmers, higher quality of compilers, better suitability for the embedded domain. 
This approach, however, has been criticized by the C++ community (\cite{stremcpp}), specifically for the inability of the 
limited language to take advantage of the C++ standard library, which requires the C++ language features, absent in 
Embedded C++.

The second approach to modify a language, to get it more suitable for the embedded development, consists of extending 
the language with constructions specific to the domain. Such approach is taken, for example, in the \gls{mbeddr} 
(\cite{2012_voelter_mbeddr_extensible_c_based_language_and_ide_for_embedded}), for C programming language. 
Extensions to C language developed in \gls{mbeddr} include state machines and decision tables.

A special language engineering environment can be used to support modular and incremental language development, e.g. \Gls{mps}.
A language is split in a special class-like items, called \glspl{concept}. As an example of a concept expression can be taken. Over 
the inheritance mechanisms, it is possible to extend languages, providing new concepts. For example, expression concept can
be extended to support new sort of expressions. Building a general purpose programming language in a language engineering
environment brings a basis to develop domain specific extensions to a well-known general purpose language.

Additionally to the language modification, the integrated development environment (\gls{ide}) can 
be improved to support domain specific development. Various analyses can be built in into the code 
editor in order to detect inconsistencies, or, simply, ``dangerous'' constructs, and inform the programmer.
Certain code formatting, or standard requirements could be enforced as well, and many more.

% Problem 
A mixture of the three techniques could be used in an attempt to achieve a ``better'' C++ for embedded development.
A special \gls{ide} can be created together with a new C++ language flavor, which prohibits the most ``dangerous'' C++
constructs, and allows for modular creation of extensions.

Before the chance to create extensions to the C++ language, the C++ language itself together with a special \gls{ide} for it
has to be created. This is the problem to be solved by this Master Thesis.

% Approach
The approach taken in this work goes even further into exploring the language modularity. While building the \cpppl\ itself 
with the goal of extensions for it in mind, the C++ itself is being build as extension to the C language itself.

Although the C++ is a separate from C language, the high degree of similarity allows to make use of the C programming language,
implemented by the \gls{mbeddr} as a foundation.

This work explores further the support, provided by \jbmps\ for the modular language construction, \cite{2012_ratiu_modular_dsls_and_analyses},
and reviews it from the architectural point of view.

The \cpppl\ is provided with a number of analyses for it. The question of analyses run-time and complexity is discussed.


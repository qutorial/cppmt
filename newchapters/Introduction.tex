\declchap{Introduction}{intro}

\section{Context}

In embedded programming the C++ programming language is widely spread, \cite{embedlangs}. Being a general purpose 
programming language, C++ does not provide, however, any special support for preogrammers of embedded systems. 

By changing  the language itself, together with a tool set for it, it is possible to get a better environment 
for a dedicated domain, for example, specifically for embedded programming. There are two known approaches to change
the language itself.

\paragraph{Taking a Subset of a Language}

The first possible approach resembles dropping some language features, to get the language, which is simpler. 
As an example, a subset of C++, called \Rg{emcpp} can be brought, \cite{emcpp}. The approach taken in \Rg{emcpp} is 
omitting very many core features of C++: virtual base classes, exceptions, namespaces and templates. 
It allows for a higher degree of optimizations by compiler possible and makes the language much less sophisticated
in general.

\Rg{emcpp} was intended to ensure higher software quality through better understanding of the language
by programmers; higher quality of compilers, through simplicity; better suitability for the embedded domain, through
memory consumption considerations \cite{stripepp}. 

The C++ community has criticized the approach taken in \Rg{emcpp}, specifically for the inability of the 
limited language to take advantage of the C++ \rg{stl}, which requires the C++ language features, absent in 
\Rg{emcpp}, \cite{stremcpp}. As a response to it IAR Systems have developed \rg{exemcpp}, which includes many of the language features,
omitted by \Rg{emcpp}, and a memory-aware version of \rg{stl}, extending not only \rg{emcpp}, but C++ in general \cite{extendedembeddedcpp}.

\paragraph{Extending a Language}

The second approach to modify a language in order to get it more suitable for a specific domain\footnote{Here we consider an 
improvement to a language to get a better suitability to any given domain, and not necessarily just embedded development.
However, we have in mind the embedded domain as first to support, as \mbdr\ targets it directly.} consists of extending 
the language with constructions, specific to the domain. The authors of the \mbdp\ have taken such approach, to improve on
the C programming language, \cite{2012_voelter_mbeddr_extensible_c_based_language_and_ide_for_embedded}. 

Specific extensions may represent some often met idioms in the domain. For example, in embedded development a 
specification for a device might be given. The goal of the programmer could be to develop a software, which 
interacts with the device. The device specification may contain a state table or  a state machine diagram,
describing the device behavior or the way to control the device. A language developer could incorporate such 
originating from the domain of interest notions into the language.

For example, the \fig{dectabtaken}
demonstrates\footnote{The illustration is taken from \cite{2012_ratiu_modular_dsls_and_analyses} }, 
how a decision table is built in into an \mbdr\ C code. The decision table provides a higher abstraction level, 
when compared to the existing initially C language constructs. 

\gr{dectabtaken}{Example of a Decision Table, Added to C Language}{0.8}

Moreover, higher-level extensions could induce some higher-level semantics. An \rg{ide} under construction could check 
the higher-level semantics for correctness on the programming stage. Such checks could improve 
quality of the software under development. For example, a \mbdr\ checks a given decision table for completeness of choices 
and their consistency, \cite{2012_ratiu_modular_dsls_and_analyses}. The \fig{dectabtaken} demonstrates a decision table. 
The decision table takes \cc{mode} and \cc{speed} as input parameters and returns a new \cc{mode} value as determined by the 
input parameters. A careful reader could see, that the exact value \cc{30} of \cc{speed}
is not taken into account, and the default value \cc{FAIL} is going to be returned. A programmer could invoke an
analysis\footnote{The analysis is described in detail in  \cite{2012_ratiu_modular_dsls_and_analyses}}, to find out that the 
table is not covering the whole choice space.

\subsection{\mbdr: a Language Engineering Project with JetBrains MPS}

The \mbdr\ development team has used a special language engineering environment, \Rg{mps}, to support modular and incremental 
language development.  A programmer using \jbmps\ splits a language under development into special class-like items, 
called \rgp{concept}. Concepts represent the \rg{ast} node types. 

As an example of a \rg{concept} an expression can be taken. It is possible to describe in \Rg{mps} different 
expression kinds, similar to object-oriented class hierarchy, allowing the objects to reference each other, 
and enabling polymorphism, in a way when any descendant can be used instead of its ancestor, e.g. 
binary minus expression can be used wherever an expression (any expression) is required. 
After various expression types were described to the language engineering environment as \rgp{concept}, 
the environment provides a chance to instantiate concrete expressions and edit them, acting as an editor for 
the language. The created editor serves as a key component in the \rg{ide} under construction. This editor
contains the code for automations and analyses, if provided for the language. It works with text code generators, 
a compiler, a debugger and all other tools connected to the language. Thus we often use the term ``editor'' and 
the term ``\rg{ide}'' interchangeably.

Over the inheritance mechanisms, it is possible to extend languages, providing new \rgp{concept} as descendants of 
the existing ones. For example, the expression \rg{concept} can be extended to support a new sort of expressions, like decision tables.
Thus, language modularity is achieved and incremental development is made to be possible.

Modularity is achieved as well, since one language is enabled to interact with another one. For example, expressions,
described independently as a language, can be reused in any language, which has a need in expressions, like a language with
statements of a programming language, because statements include expressions naturally.

Having in mind the opportunities, the language modularity in \jbmps\ brings, it makes sense to recreate a general 
purpose programming language in \jbmps. Building the general purpose programming language brings a basis to develop 
domain specific extensions to the well-known general purpose language. The editor for the general purpose language comes 
almost ``for free'', as a side product, after the language is split into \rgp{concept}.

Later, from the code in the implemented general purpose language a text code can be generated for further processing, 
compiling, deployment. The language extensions, of-course, are not known to the existing tools which process the language.
But they usually can be reduced to the base general purpose programming language statements, presenting the regular 
syntax of the taken general purpose language to the further tool chains, which expect only the basic language constructions, 
as an outcome.  Thus the general purpose programming language is getting enhanced, remaining compatible with all the 
existing tools to process it. The developed editor forms a key component of a new \rg{ide}, as described above.

Additionally to the language modification itself, an \rg{ide} can be improved to support the domain specific 
development. Various analyses\footnote{analyses not only for extensions, but for the base language itself} 
can be built in into the code editor in order to detect inconsistencies, or, simply, ``dangerous'' constructs, 
and inform the programmer. Certain code formatting, or standard requirements could be enforced as well. 
The \rg{ide} could be enhanced with various automations, like support for code generation and refactorings. 

As the new \rg{ide} works internally with an \rg{ast}, described through the node types, or \rgp{concept}, in order to perform a code
analysis, a generation, or a transformation, there is no need to invoke parsers for the code, which is advantageous.

\section{Problem}

Being a powerful extensible tool for an embedded systems developer, \mbdr\ does not support the \cpppl. Supporting C++ would be an
advantageous argument for \mbdr, as C++ empowers the developer with additional paradigms\footnote{C++ represents 
a multi-paradigm programming language, as functional programming or programming with templates can be seen as 
distinct paradigms. C, in comparison, is a single-paradigm language, with the only one procedure-oriented paradigm supported.}
of programming, mainly the object-oriented programming. Thus a problem appears, to support C++ within \mbdr. As \mbdr\ is built itself  upon
\jbmps\ and language modularity principles, these principles have to be taken into account while extending \mbdr.
This means, that the \cpppl\ has to be developed as a modular extension for the C language provided by \mbdr.


\declsec{Approach}{approach}


Above we describe the two approaches used to make a language more suitable for a particular domain. 
In this work we use a mixture of the two approaches in an  attempt to achieve a modular C++ language, 
which can later on serve as a basis to target some domain of interest. We try to modify the C++ language 
for it to be, on the one hand more, user-friendly, and, on the other hand, for it to be prepared to become 
a better embedded systems development tool, being safer, clearer and, in the future, including specific
for the domain extensions. We built C++ itself as an extension to C and as a modular base for further extending. While limiting 
C++, we try to keep all the core features in it, to not to face the same problems as \Rg{emcpp} had.

We built C++ as a suitable base for the further language engineering, including the specialization  of C++ for embedded development, 
and even more general, for any domain of choice. A special \rg{ide} is created together with a new C++ language flavor, which supports 
a C++ programmer.

During the creation of the \cpppl\ in the way described, the language modularity in general is analyzed, and caveats of it
are described together with the ways to avoid them. The newly created \rg{ide} features analyses. The question of their
computational complexity is raised in general, together with the practical outcomes of it.

The approach taken in this work goes further into exploring the language modularity on the basis of \jbmps. While building 
the \cpppl\ itself  with the goal of embedded domain specific extensions in mind, the C++ itself is being built as 
an extension to the C programming language, provided by \rg{mbeddr}. The C++ flavor implemented in \jbmps\ and discussed in this 
work we call \pcpp. Although C++ is a separate from C language, the high degree of similarity allows to make use of the C programming language,
implemented by the \rg{mbeddr} as a foundation. Not only reuse of the basic C is achieved, but also the embedded extensions from
the \rg{mbeddr} are immediately supported by the newly built C++.

The ultimate goal during the reuse of \mbdr\ as a base for C++ is keeping \rg{mbeddr}
  not modified towards the \cpppl\ \emph{only}, but instead, making, when needed, \mbeddr\  more extensible \emph{in general}, 
so that both resulting C++ and the base for it, \mbdr,  can develop further being disjoint to a high degree.
This independence of \mbdr\ from the C++ extension ensures, that the \mbdp\ can develop further without looking back 
on C++, making the \pcpp\ support an independent task.

\section{Contribution}

In this work we describe a \cpppl\ implementation\footnote{This implementation does not represent complete C++, and
limitations are discussed along this work.} on top of \mbdp. Our contributions towards it are listed below.


 \paragraph{Pure modular extension for \mbdr.} The task of a one-side-aware only extension is a challenge for the whole language modularity concept, 
provided by \jbmps. This work explores further the support, provided by \jbmps\ for the modular 
language construction, c.f. \cite{2012_ratiu_modular_dsls_and_analyses}, and reviews it from the architectural point of view,
summarizing finally the support for it, provided by \jbmps.

 \paragraph{Improved C++ programmer support.} This work contributes the \cpppl\ with a number of automations. The automations include code 
generation and structuring. They are designed to compensate on some caveats of C++, or lack of support for 
several aspects in the language itself, like a support for a coding style. 

 \paragraph{Improving on C++ language itself.} C++ is known for a number of pitfalls, 
 a programmer can be caught by. This work tries to improve on this situation by introducing analyses.
The analyses are intended to increase understanding of a constructed code by a novice programmer, 
or to provide information to an experienced C++ professional. Analyses together with automations are provided to 
achieve an improvement in quality, security and understanding of the pure C++ code. The automations and analyses 
are mostly implemented as a programming on an \rg{ast} in a Java-like programming language.

 \paragraph{Analyses run-time and complexity problem researched.} As analyses and automations grow in complexity and quantity, the question of their computational complexity arises.
In the \jbmps\ \rg{api} it is not explicitly defined, when analyses provided for a language take place, 
how much of a computational resource they can take advantage of, and how an end user should be informed on results and a progress. 
These aspects may affect the overall \rg{ide} behavior, including performance, as the analyses complexity may be high and the results of them could 
be of a high value. The question of analyses run-time, complexity and results presentation is raised and discussed in 
this work in general and in particular, suggesting improvements to \jbmps\ \rg{api}.

 \paragraph{General language recreation principles.} Generalized principles, useful when recreating an existing language in a projectional 
 environment, were formulated and supported by practical examples in this work.
 
 \paragraph{Language extensibility in \jbmps\ researched} Being one of the largest extensions to a language ever constructed in \jbmps, \pcpp\ 
 represents a good basis to discover general language extensibility support limitations of \jbmps. The support for extensibility in \jbmps\ is 
 reviewed considering each view on a language, workarounds for current \jbmps\ version are offered, and improvements for future \jbmps\ development
 are suggested.
 
 \vspace{10 mm}
 
Finally, it is fair to say, that the \mbdr\ team\footnote{especially Markus Voelter} have already grounded some practical foundations for the \pcpp, 
before the start of this work. Some of them, were kept either as suitable (reference type) of for the further improvement (templates) and just described and analyzed 
here, some of them were considerably reworked (classes, inheritance, encapsulation and polymorphism). 

Couching from the \mbdr\ team made us changing some of the implementation aspects to be different from what we planned originally (namespaces,
operator overloading partially). And, of-course, we built some parts new, from scratch, without any influence (at least at the moment of being) \mbdr\ team 
at all (all analyses, construction and copying, some more). 

The evaluation of the experience, gained by us during the practical implementation part, results into the extensibility analysis, 
research on complexity of checks and the run-time for them together with suggested \jbmps\ improvements, and general guidelines 
for building projectional language implementation represent my own theoretical focus and commitment.

\section{Structure of the \MT}

In the \rchap{found} we describe in general two approaches, \rg{ide} developers can take when
building a new \rg{ide} for a language, together with the language itself. We describe the traditional 
textual approach followed by the newer \rg{projectionalapproach} used in this work. The language 
modularity and the way a language is described in a projectional environment are discussed. Finally,
we shortly touch the problem of importing an existing code base into a projectional \rg{ide}.

The \rchap{techno} is dedicated to the main technologies used in this \MT. At first an environment, providing
all the facilities for building a projectional \rg{ide} is described, \jbmps. Then, we describe the \mbdp\
which serves as a modular basis for the present work practical achievements. Last, we give a number of 
references for the reader, who wants to get a better level of familiarity with the \cpppl.

In the \rchap{cpp} we discuss the practical questions of the \pcpp\ implementation. At first, we align,
why the \mbdr\ C implementation can serve as a good basis for this work, and describe primitive 
extensions towards C++ for it. Next, we discuss all language features of C++ related to the 
object-oriented programming, and their implementation in the \pcpp. After that, we describe the implementation
of operator overloading and templates, as advanced C++ features. And finally advanced \rg{ide} functionality
is described, including analyses and checks, supporting the C++ development.

In the \rchap{eval} at first we compare the \rg{projectionalapproach} and the textual approach. Then we list and describe
the generalized principles, which could be used, when creating a projectional \rg{ide} for an existing language. Next, the 
\jbmps\ support for language extensibility is reviewed. Later, we discuss analyses and their development problems, in a light of 
their computational complexity. Finally, briefly the outcomes of C++ concepts approach to C++ templates is evaluated.

The \rchap{concl} concludes the \MT\ and suggests the potential future work.



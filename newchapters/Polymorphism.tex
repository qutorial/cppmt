\subsection{Polymorphism}

There are several ways to achieve a polymorphic behavior in C++. Purists of the language differentiate 
between the polymorphism based on virtual functions or based on templates. More general opinion can include
the notion of the polymorphism also functions overloading and operator overloading, also called 
\emph{ad hoc} polymorphism, polymorphic solution for a single concrete purpose. 
Occasionally, various operations with the \cc{void *} type are also classified as a polymorphic programming.

In this section we are writing only about the class-related virtual functions polymorphism, and the way it is 
implemented in \pcpp.

\subsubsection{Virtual Functions Polymorphism in C++}
\label{cpppolydefs}

At first, we describe the polymorphism, as it is implemented in the \cpppl, pointing out the places, where it could be improved.

Dislike many other popular object-oriented programming languages, e.g. Java, in C++ there is no pure notion of
an interface. Instead, a base class, its public part, is used as an interface declaration for descendants. 
Functions, designated to be a part of the interface, must be declared \cc{virtual} and they can be overloaded in
the subclasses.

\cpp{Pure Virtual Function Syntax Example}{purevirtual}

The virtual functions in the base-interface class can be implemented as well, providing some ``default'',
common enough behavior. Otherwise they are left \rg{purevirtual}, meaning that no implementation is provided
and the pointer to the function in the table of pointers to virtual functions is zeroed. The syntax
for \rg{purevirtual} functions is rather not obvious\footnote{Especially when not knowing about the zero pointer value semantics.}
and a bit cumbersome requiring to type one reserved word, one punctuation sign and one digit to 
express a simple fact of pure virtuality or, simply, absence of implementation, see the \rl{purevirtual}.


The approach of classes with virtual functions as interfaces is more flexible compared to languages with the notion 
of an interface directly introduced.
In C++ it is possible to create partially implemented base classes, what can not be done when implementing
an interface in Java or C\#, where the approach is ``all or none'' regarding the implementation of interface
functions. The presence of interfaces in the language can be though considered a more clean way to program.

In order to implement a declared in a base class function, a descendant must declare and implement
a virtual function, matching the full (including a return type) signature of the declared in the base class function, see the \rl{override}. 
The connection to the ``interface'' function declaration stays subtle however. It is not immediately clear, whether the 
declared new function in the descendant is an override of an existing in the base class function or an independent declaration
of an entirely new function in the descendant class. This knowledge affects the changing process greatly, as the 
override should change from the interface, together with all the implementations. The absence of a clear, 
explicit override syntax we call here an ``\rg{ovsa}''.

Whenever a class and all of its ancestors do not provide an implementation of a certain virtual function,
created as a \rg{purevirtual} in the declaring class, the class is called an \rg{abstract} class. It is not possible
to construct instances of an \rg{abstract} class. C++ however does not have any special syntax to explicitly declare a 
class \rg{abstract},  see the \rl{purevirtual}.
The programmer usually has to be aware (from documentation, implementation, or, the worst case, compilation
errors) whether a given class is \rg{abstract}. We will call this phenomenon an ``\rg{acsa}''.

Overriding a function is an active action of the programmer, and it is initiated by the 
programmer. I.e. the programmer wants to state, that a new function is designed to override an existing one, and 
which is the overridden function. The abstract property of a class, oppositely, is not a quality a programmer directly
gives to a class in C++. It can be rather deduced from the analysis on the base classes automatically, by editor. So in this case
no actions are needed from the programmer side. Because of this, the two similar, at the first glance, absence of syntax 
phenomena are resolved or differently from in the \pcpp\ implementation.

In order to use an interface, declared in some class, the using code has to get a pointer to an instance
of any inheriting the interface descendant class instance. Thus, the type system has to allow a pointer
to the descendant to be treated in the way as a pointer to the base class would be. The same should hold, 
normally, for the reference types, but it is not used very often in practice, and is omitted in the implementation.

As this typing rule represents the core of polymorphic behavior, we will start from it below, describing further the
polymorphism in the \pcpp\ implementation.

\subsubsection{Special Typing for a Pointer to a Class}
\label{pointertoclasstyping}

The problem solved here is enabling the usage of pointers to descendants instead of pointers to ancestors, see the
\rl{polycall}.

\cpp{Example Usage of a Pointer to Descendant Class}{polycall}

The \mpsid{PointerType} \rg{concept} is implemented in the \mpslang{pointers} language in \mbdr. 
Following the goal of non-invasive changes to \mbeddr\ the \pcpp\ implementation needs to add the 
typing rules for pointers \emph{to classes} without changing the  \mpslang{pointers}  \mbdr\ language, where the 
typing system for the pointer type is defined.

In the case when a type of a pointer to a base class is expected, a pointer to a subclass should also be accepted, like
in the \rl{polycall}. The type system of the pointer language will try to check the compatibility of the two pointer types. It will
fail to do so, as the \mbdr\ languages are not aware of the \pcpp\ extensions\footnote{The \mpsid{ClassType} \rg{concept} in this case.} 
by design.

In a case like this a replacement rule can be used, see the Section \ref{mpsts}. \pcpp\ provides such a replacement rule for the 
pointer type, which checks, whether a class pointed to, is a passing descendant of the class required by an expression, 
where the pointer was used, and performs the necessary typing.

In the \rsec{extensibility} we discuss more on the approach taken here, and its limitations.

\subsection{Safer Casting for a Pointer to Class}
\label{safecasting}

\cppproblem

The type casting for class types in C++ has a number of disadvantages.
\begin{itemize}
 \item The \cc{const\_cast} is usually a signal of a design error, since the 
 declared \cc{const} property of a value is taken away.
 
 \item The \cc{static\_cast} allows for wrong conversions from a base class to a derived class,
 which could cause a failure in the run-time, as no checks are performed.
 
 \item The \cc{reinterpret\_cast} allows to ignore the possibility of  
  a completely meaningless conversion, which can lead to run-time crashes, neither run-time nor 
  compile-time checks are performed.
 
 \item The \cc{dynamic\_cast} has usually low compile-time support, and relies on the run-time type
  information to perform checks at the run-time. This information may not be available, being,
  for example omitted for the sake of performance gains.
\end{itemize}


\pcppsolution

A new construction for type conversions is introduced, which can be considered a slight improvement
over the existing in C++ casts. The \mpsid{as} construction translates to the C++ \cc{dynamic\_cast}.
Additionally, checks are added to improve the security of the construction.

\ms{nonpolycasting}{Example of the \mpsid{as} Construction}

The \fig{nonpolycasting} demonstrates some of the introduced in \pcpp\ features related to the class type conversions.

The new \mpsid{as} construction is being checked in the coding-time. For example, the cast from \cc{parent} to 
\cc{NPChild\*} is signaled as an error, since these class hierarchy is not polymorphic, and the \cc{dynamic\_cast}
in the basis of \mpsid{as} would not work on it.

The next cast from \cc{parent} to \cc{NonPoly\*} is signaled with warning, because the \cc{parent} pointer does 
\emph{already} have the requested type. Such casting may signal a lack of understanding by the programmer while coding.
The \mpsid{as} construction returns \cc{0} at the run-time when conversion fails. Thus a good programming style
should include a check for it at the run-time.

When \mpsid{as} is used to cast a polymorphic class, it checks, whether the cast makes sense.
For this the related classes must be in the same class hierarchy. When a zero result is possible,
the node is highlighted with warning, for the programmer to take an action of checking. Casting 
between unrelated classes is presented as an error.

Some implicit type conversions are always meaningful, thus a conversion from a pointer to 
a child class to a pointer to a parent class is recognized by \pcpp\ and is allowed, as 
described in the Section \ref{pointertoclasstyping}.

The casting can be further improved through introducing new analyses. At first, a fast type of cast, 
like \cc{static\_cast} or even \cc{reinterpret\_cast} can be used in translation, when a data-flow analysis
can prove, that the conversion is always meaningful. Secondly, unnecessary casts can be eliminated in this way.


\subsubsection{Overriding a Virtual Function}
\label{overridefunction}

The \rl{override} demonstrates, how overriding of a virtual function happens in C++.
The function \cc{getArea()} is initially defined in the class \cc{Shape} and is then overridden
in the class \cc{Circle}. The \rl{override} demonstrates as well the \rg{ovsa} in C++.

\cpp{Example of an Overridden Function - Text Code}{override}

In the \fig{override} the same example is demonstrated, but in \pcpp.
In fact, the code shown in the \rl{override} can be generated 
from the demonstrated projectional sample in the \fig{override}, by invoking all \Rgp{textgen} in \jbmps.

When a method declaration is created, it can be set to be an override of a method in a base class.
Right after an empty method declaration is created, it can be set to be an override, so that the only
thing which stays, is to pick a method from a base class to be overridden. The override is automatically
named, the parameters and the return type are set accordingly, the \cc{virtual} property is immediately
set. This is yet another work saver for a programmer.

After the override has been linked to the overridden method, the projectional editor checks, if 
the override full signature stays precisely the same as the one of the overridden method. Thus if the 
latter changes, it is going to be indicated in the former.
The overridden method is shown next to the override declaration, see the \fig{override}, which 
compensates on the \rg{ovsa}.

\ms{override}{Example of an Overridden Function - \pcpp}

The additional, not belonging to C++, syntax in \pcpp, should not confuse the reader. It is only 
present in the projectional editor, and, when an \rg{ast} is generated into a text code, a regular
C++ syntax is achieved. However in this case an \rg{ast} stores \emph{more} than needed to 
generate and compile a C++ code. 

This subsection is one of the examples, that storing more information can be useful,
it is generalized in the \rsec{genprinciples}.


\subsubsection{Pure Virtual Functions}
\label{purevirtualfuncs}

In the example code in the \rl{override} one improvement to could be made. As the \cc{getArea()} for a random shape can not 
be determined, it makes sense to make the \cc{getArea()} function \rg{purevirtual}. As said before, a \rg{purevirtual} 
function has no implementation in the declaring class, and serves only the overriding purpose. Semantically in C++ it sets the 
pointer in \cc{vptr} table to 0 and thus has reflecting the zeroing syntax, see the similar example in the \rl{purevirtual}.
This syntax can be seen as not obvious, as it reflects more the under-the-hood implementation of the mechanism,
rather than the original programmer intention to built a basis for an override chain, while taking the advantage of 
polymorphism.
Additionally, as discussed above, declaring a \rg{purevirtual} function requires a significant amount of the syntactical
overhead. 
These were the disadvantages of the \cpppl, \pcpp\ tries to improve on.

In the \pcpp\ implementation, one \rg{intention} is reserved to make a function \rg{purevirtual}. The \rg{intention} automatically
sets the needed pure virtual and virtual flags of the function declaration, and the projection changes. That is why out of the 
statement ``\cc{virtual double getArea() = 0;}'' a programmer has to input only the name \cc{getArea}, pick the type \cc{double}, and
toggle the ``Make pure virtual'' intention for the declaration in the \cc{Shape} class.
The result of the intention work can be seen in the \fig{purevirtual}.

\ms{purevirtual}{Example of a Pure Virtual Method - \pcpp}

The word \cc{pure} is added by the projectional editor. This makes the reading of the code easy and natural.
The ``\cc{ = 0}'' part is preserved for the C++ programmer with habits, used to the original C++ syntax. And, as in many other cases, 
the semicolon is omitted as it is nothing but syntactic help to the compiler, which does not have to appear 
in the projection at all.

This example demostrates, how the lower-level syntax can be made more readable. The general principle on it is formulated in
the \rsec{genprinciples}.

\subsubsection{Abstract Classes}
\label{abstractclasses}

If any of  \rg{purevirtual} functions, in an inheritance chain, leading to a class, is not implemented by this chain, 
or inside the class itself, the class is called an \rg{abstract} class. It is not possible to construct an instance of an
\rg{abstract} class, as such classes have not implemented methods.

A programmer has to know which classes are \rg{abstract},
and are intended for inheritance and further implementation only. One of the reasons for this, is to not to try instantiation of
an \rg{abstract} class. Another reason is to exactly identify \rg{abstract} classes, designed to serve as extension points.
Above we discuss the \rg{acsa} in C++. It leads to the need for the programmer to determine somehow him/herself 
if a given class is \rg{abstract}, the source code representation of a class does not give any information on it, 
unless some naming conventions require \rg{abstract} classes to be named specially.
An improvement to this situation would be a behavior, when an editor can perform an analysis and 
determine if a class is \rg{abstract}, hinting on it to the user. In the case of the projectional
editor, this analysis is especially computationally efficient\footnote{In comparison to textual editors, with the need to 
parse, see the Section \rsec{comparison} on comparison to the textual approach, and \rsec{analysesandcomplexity} on complexity of analyses.}, 
as an \rg{ast} is readily available, and a quick  analysis can be performed by a simple recursive algorithm on inheritance chains.
%TODO Make an appendix with the algorithm and reference it

After an \rg{abstract} class has been determined, it is possible to modify the editor representation for it, 
and show that it is \rg{abstract}, see the \fig{abstract}.
In this example a typical class hierarchy is created to support user interface programming. A user of this \rg{api}, when 
searching for a button, could try using the \cc{Button} class, which is designed to be \rg{abstract}, and serve as a 
base for the further implementations, e.g. \cc{PushButton} in the example, or check boxes, radio buttons and similar,
which could be alternatively created.

\ms{abstract}{Determining Abstract Classes}

The projectional editor, however, checks on the fly, if a certain class is \rg{abstract}, adding a special \cc{abstract} word 
in front of its declaration. This makes the reading easier, and allows for quicker understanding of a code. 

This example demonstrates a general principle, see  the \rsec{genprinciples}, of an advised practice to 
perform quick analyses and inform the user.

Additionally, the creation and usage of \rg{abstract} class instances is checked, and forbidden by type analysis. This is 
described separately in the \rsec{advanced}.

\declsec{Operator Overloading}{operatoroverloading}

\cppproblem

The operator overloading feature in C++ allows to redefine semantics for a given operator when used 
with certain types. Indeed, this represents a language engineering methods, when some language construction 
(operators) are extended to support new features (work with new types). The projectional way to support such
needs would be extending the language with new construction. However, moving towards the \rg{stl} support, 
it is good to provide the operator overloading usual for C++.

Most often, the operator overloading feature is used together with class types. A very common examples are iterators with 
their increment, equality and dereferencing; standard input and output streams with stream operators \cc{<<} and \cc{>>}; and \cc{std::string}
concatenation with operator \cc{+}.

\cpp{String Concatenation with Overloaded Operator +}{strings}

In the \rl{strings} one can find 4 overloaded operators: 
\begin{itemize}
 \item for \cc{std::string} output,
 \item for \cc{const char*} output,
 \item for \cc{std::endl} output, and 
 \item for \cc{std::string} concatenation.
\end{itemize}



\pcppsolution


\gr{operatoroverloaddecl}{Operator Overloading: Definition}{0.3}

We will discuss the implementation of the operator overloading in \pcpp\ based on the 
example. The \fig{operatoroverloaddecl} shows (partially) a definition of the \cc{Coords} class.
The class represents a vector in the two-dimensional space. Three operators are overloaded for it:
\begin{itemize}
 \item \cc{operator +} performs the addition of two vectors,
 \item \cc{operator +} subtracts one vector from another one,
 \item \cc{operator []} allows to reference x and y coordinates of the vector in the array fashion.
\end{itemize}

Other members of the \cc{Coords} class should be self-describing, given the vector semantics.

Inside the class the overloaded operators get declared. A reader can notice a high degree of
similarity between an operator overload declaration and a method declaration, as both 
have a return type and arguments. Thus a common base \rg{concept} was created, 
\mpsid{AbstractImplementableAsMethod}. It allows to link implementation to it, or a method
definition, in C++ terminology.

The difference between a method declaration and an overload declaration, is that the
overload declaration has an operator designator instead of a name. An abstract \rg{concept}
\mpsid{OperatorDesignator} has been created to serve as a base for all possible operator designators
for overloading, like \cc{+}, \cc{-}, \cc{[]} in this example and more.

Creating a concrete operator designator, and inheriting it from the abstract \mpsid{OperatorDesignator},
makes it possible to define a new operator for overloading. Thus declaration and 
definition of an operator overload task is resolved in a modular way.

Left, stays the way, overloaded operators can be incorporated into the existing
hierarchy of expressions.

\ms{operatoroverloadingusage}{Operator Overloading: Usage}

The \fig{operatoroverloadingusage} demonstrates, how the defined operator overloads could 
be used. The \cc{assert} statements are taken from the special \mbdr\ language designed to
create test cases. They test the condition and if it is \cc{false} the execution stops.

At first, in the example two \cc{Coords} objects are created. Next, a third object is created 
as their sum. The \cc{+} operator used is nothing else but an instance of the \mpsid{PlusBinaryExpression} from 
\mbdr. Next, the newly created \cc{v3} object gets tested. After that, the \cc{v4} object is
created as a difference between \cc{v2} and \cc{v1}. Again, the operator \cc{-} used to calculate 
the difference is an instance of the \mpsid{MinusBinaryExpression} in \mbdr. Then, \cc{v4} values get tested. 
And finally, the \cc{[]} operator is used with \cc{v4} to get its second coordinate. The operator
used is an instance of the reused from \mbdr\ \mpsid{ArrayAccessExpression} \rg{concept}.

The reuse of existing expressions from \mbdr\ is highly beneficial. 

For example,  the \mpsid{ArrayAccessExpression} features a complex behavior. At first the indexed 
object is typed as an expression. After that, after \cc{[} gets typed by the programmer,
the indexed object is transformed into an instance of the \mpsid{ArrayAccessExpression}
\rg{concept}. The \mpsid{ArrayAccessExpression} has two children. First is the expression
which corresponds to array. The indexed object is moved to be a child of the new \mpsid{ArrayAccessExpression}.
The second child is an expression, which represents the index. It gets the input focus,
after the replacement discussed above happens.

Next, if the indexed expression is about to be deleted, a special action\footnote{See \ref{othermpsinstruments}}
takes place, which replaces the \mpsid{ArrayAccessExpression} with its child with the role\footnote{See \ref{mpsconceptdeclaration}}
\cc{array}. Thus the deleting of index resembles the usual behavior as in text editors, where
just the index disappears, not the whole expression together with the indexed object.

The decision to not to create special expressions, which would represent the usage of
the overloaded operators, helps to avoid code duplication, reprogramming the behavior of
existing expressions in \mbdr. 

Alternatively, the expressions in \mbdr\ could be inherited and changed. But the 
degree of changes, possible through inheritance is limited, as described in \rsec{extensibility},
and creation of a second set of the same looking expressions could potentially confuse the
programmer, when he or she is going to instantiate one of the many \cc{+} operators, for example.

The existing expressions could not be reused as-is, however. Their type system, specified
in \mbdr, is not aware of the \mpsid{ClassType}, introduced by \pcpp. Thus the class
objects would get rejected by the type system, as the type for the resulting expression
would not be figured out be \mbdr.

Changing the \mbdr\ type system, hard coding the necessary behavior for classes, would 
introduce a dependency of \mbdr\ code on the \pcpp\ code. The goal of this work, however,
is to avoid such situations, as described in the \rsec{approach}.

The need to extend \mbdr\ in order to allow correct typing for the \mpsid{ClassType} in
existing behavior was necessary.

The solution for this problem was found to be as follows. For each of the expression kind 
in \mbdr\ an \rg{interfaceconcept} could be introduced, which will participate in the 
type system for the expression, and, if found to be one of the expression parts, 
be delegated with the task of the type calculation. Then, the \mpsid{ClassType} is changed
to implement the newly create \rgp{interfaceconcept}, and the code is added to it, to 
compute the typing. The \mpsid{ClassType} looks up the corresponding \mpsid{Class} for 
the presence of overloads, and if an overload for the operation in question is found, 
the overloaded operator return type is taken as the result, otherwise an error is reported.

For example, the \mpsid{ClassType} implements a newly created in \mbdr\ concept, 
\rg{interfaceconcept} \mpsid{ISelfTypingInBinaryExpression}.
The \mpsid{ClassType} implements this \rg{interfaceconcept}. Thus, when a type of a 
binary expression is calculated, the \mpsid{ClassType} is delegated with the task to 
determine the possibility to be involved in the binary operation under question, 
and to calculate the resulting type.

In the \fig{operatoroverloadingusage} when calculating \cc{v3}, for example, the 
\mpsid{PlusBinaryExpression} delegates to the \mpsid{ClassType} corresponding to \cc{Coords} 
class, to figure out, if the operation is possible. According to the declarations in the
\fig{operatoroverloaddecl} the \cc{Coords} class has an overload for the \cc{+} operator,
the argument type is \cc{Coords} class, thus the operation is determined to be possible,
and the return type of the operator, namely \cc{Coords} class, is taken as the result
type for the \mpsid{PlusBinaryExpression}.

As the \mpsid{ISelfTypingInBinaryExpression} \rg{interfaceconcept} is declared in
\mbdr, the only dependency created by using it, is the existing dependency of the \pcpp\
on \mbdr, keeping \mbdr\ development separate from the \pcpp.

The current \pcpp\ implementation has some limitation in the operator overloading:
\begin{itemize}
 \item Not every operator is ready for overloading, and \mbdr\ has to be further extended 
 to support it, this is, however, not an entirely new task, but rather repetitive application
 of the approach described above.
 \item C++ features not only member operators, but also static operator overloads, this is 
 not supported by \pcpp\ in the current version, a very similar to member operators approach can be 
 taken to implement it.
\end{itemize}

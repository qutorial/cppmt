\declsec{mbeddr Project}{mbeddr}

The \mbdp\ is a software built with the use of \jbmps.

The \mbdrp\ represent mainly an implementation of the C programming language in the \jbmps\ environment. Having embedded systems
and software for them as the main focus, \mbdr\ provides certain language extensions to empower the programmer in the mentioned domain \cite{mbeddr-wave}, 
\cite{Voelter:MoDELS:2010}. 

Being a different language the \cpppl\ shares a lot of commonality with the C programming language. 
As \jbmps\ allows, to some extent, see \ref{modularity}, incremental language construction, the \mbdrp\ represents 
a suitable basis for the \cpppl\ implementation in \jbmps\ . 

We could not  extend \mbdr\ with C++ pure incrementally. We had to perform some changes to \mbdr\ itself. 
The changes were introduced however, in a way to make \mbdr\ more extensible in general. Compare this to creating 
a separate branch of \mbdr\ designed only for C++. After such forking a separate team would have to support a newly
created \mbdr\ version for C++. By making \mbdr\ more extensible, and by building C++ as a pure extension we keep
only one \mbdr\ version, and reduce the maintenance needed in future, to keep C++ supported be \mbdr.


In this section we describe the \mbdp, as the current work is based on it.

No matter the \mbdp\ has  (one) C language with extensions as an outcome, internally, as a \jbmps\ software
it is represented as several \jbmps\ languages.

All \mbdr\ languages are named starting from \mpslang{com.mbeddr.} name part. In this document we usually omit it,
keeping the last word of the name only. E.g. \mpslang{com.mbeddr.expressions} is called simply \mpslang{expressions} here.

\subsection{mbeddr Expressions Language}
\label{expressionslang}

The \mpslang{expressions} language contains definitions for all expressions, possible in the \mbdr\ C language.
As in object oriented programming languages \rgp{concept} of the \cc{expressions} language form inheritance hierarchies. 
\jbmps\ is capable of showing a given \rg{concept} in a hierarchy. 

\msnozoom{minushierarchy}{\Rg{concept} Hierarchy Example}

The \fig{minushierarchy} shows a hierarchy for the \mpsid{MinusExpression} \rg{concept}. In a similar way all expressions of the 
C programming language are implemented in the \mpslang{expressions} language.

Whenever there is a need in the \cpppl\ to extend the C programming language with a new expression kind, like 
object member reference, \mpslang{new} expression and so on, a point of inheritance has to be found in the 
\mbdr\ \mpslang{expressions} language to base a new \rg{concept} on it.

Additionally, the \mpslang{expressions} language defines  C language types. 

\msnozoom{typehierarchy}{\mbdr\ Type Hierarchy Example}

All \rgp{concept} corresponding to C types are based (directly or indirectly) on the \mpsid{Type} \rg{concept}.
For example, the hierarchy of \mpsid{IntType} \rg{concept} is demonstrated in \fig{typehierarchy}. 

In order to add a type to \mbdr\ C language, one should inherit from the \mpsid{Type} \rg{concept} as well.
Such inheritance automatically allows the new type to appear at all places, where a type in general can be found
in the C language or its extensions.


\subsection{mbeddr Statements Language}

\mbdr\ \mpslang{statements} language contains definitions for C language statements. The \mpsid{Statement} \rg{concept} serves as the
base for inheritance, and represent by itself an empty line, or no-statement.

In order to create a new statement, like \cc{delete} statement in C++, the inheritance should start from the \mpsid{Statement} \rg{concept}.

\ms{stexprmarked}{Example of Multiple Languages Used Together}

The \mpslang{statements} language actively uses the \mpslang{expressions} language, \fig{stexprmarked}. In the \mbdr\ code snippet
the nodes coming from \mpslang{statements} language are marked green, and the nodes, coming from \mpslang{expressions} language are
marked yellow\footnote{not marked with color is an instance of the \mpslang{Function} concept, which comes from the \mpslang{modules} language}. As the example shows, 
\mpsid{if} statement and \mpsid{return} statement are coming from the \mpslang{statements} language,
but inside they contain as children expressions. This is an example of language modularity in \jbmps, used by \mbdr.

\subsection{Modules in mbeddr}
\label{mbdrmodules}

In C (and in C++ as well) there is no clear concept of a module. The \mbp\ improves on it, defining 
modules, \cite{Voelter:MoDELS:2010}. A C module is a \rg{concept}, from which the header and the .c files are generated in mbeddr.
Flagging an object (function, variable, structure, etc.) in a module as \cc{exported} causes the declaration of the object to appear
in the header, and thus the object starts to be accessible by other modules.

An issue with the C programming language is that there is one and only global namespace. The \mbp\ improves on it 
by introducing so-called name mangling. All names of the module contents are prefixed with the module name,
when generated to the C text code. Thus the object with the same name but from different modules do
not cause a name clash.

The implementation of modules in \mbdr\ can be found in the \mpslang{modules} language. Functions are described there as well,
for the \mpsid{Function} concept example, see the \fig{stexprmarked}, not marked with colors part.

Modules are further included into \jbmps\ \rgp{model}, which correspond to one single \jbmps\ file unit.
After a programmer finishes the code in the projectional editor, \jbmps\ generates textual C code, 
according to the rules, given by \mbdr. Each model is generated into a set of textual C files.
After the files has been generated, \mbdr\ launches a compiler to get an object code.

For the purpose of the code generation and compilation, every model should have a node of \mpsid{TypeSizeConfiguration} 
and \mpsid{BuildConfiguration} \rgp{concept}. In the former, the programmer should specify all sizes for C language types, 
in the latter, the programmer should specify, how the \mbdr\ has to invoke the compiler.

\subsection{Pointers and Arrays in mbeddr}

In the \mpslang{pointers} language array access and pointer dereferencing expressions are defined as \rgp{concept}.
Similar syntax to them have overloaded operators with classes. Thus this language could also be extended by the \pcpp,
if some reuse is possible there as well. This is discussed in detail in the \rsec{operatoroverloading}.








\declsec{C and C++}{candcpp}

Initially C++ appeared to be an extension to the C language, called ``C with Classes'' \cite{cwithclasses}. Till now the high degree
of commonality can be found between the two languages. Mainly, the most of the C code is going to be valid C++ code. 

Some differences exist however, I introduce them here, together with the way they are adopted in the \pcpp.

\subsection{Reference Type and Boolean Type}

Among primitive types and operations there are two major differences relevant to practice.
They are discussed in the following subsections.

\subsubsection{Reference Type}

The reference type is basically a new construction in C++ which represent the old notion of pointer in C with a different syntax.
In fact behind the reference type stands a pointer to the base type, and it is dereferenced when needed. 

The syntax for a variable of type \cc{T} can be used with the variable of type reference to \cc{T}, or \cc{\&T} as it is designated in C++.
Thus the programmer benefits from the shorter syntax in comparison to the pointer dereferencing.

It comes especially handy when passing arguments to methods by reference, which avoids creating a copy of an object to pass by 
value, and still preserves the short syntax without pointer dereferencing, when accessing the value.

\cpp{Reference Type in a Copy Constructor}{copyconstructor}

The value of the reference type in C++ is bigger however, than just a new syntax for pointers. A constant reference to a class object 
has to be used in a constructor to give it a special meaning of the copy constructor for the class, as the \rl{copyconstructor} demonstrates.
This changes the whole semantics of the class, making it copyable. It is discussed in detail in \ref{classcopying}.

The reference type itself is implemented rather straightforward and similar to the pointer type of \mbdr. It inherits the \mpsid{Type} 
\rg{concept} of \mbdr\, as described in the \ref{expressionslang}, and has a base type as a child.


\subsubsection{Boolean Type}

% Need in C++
The \cpppl\ has a special \cc{bool} type to represent the two logical values. There is no such type in C.

%Implementation
No matter, the \cc{bool} type is not present in C, for the convenience of the user the \mb\ C implementation introduced a type \cc{bool},
which is translated to \cc{int8\_t}, together with \cc{true} and \cc{false} values, converting to \cc{1} and \cc{0} respectively.
This implementation can arguably be considered better than the original C++ \cc{bool} type present in the generated code, because the \cpppl\ standard
does not define explicitly the size of the \cc{bool} type \cite{cpp11}. This can be suitable for the embedded domain better, conforming to the
\mbdr\ spirit.

%TODO No Goal 1
In the context of embedded systems, which \mbdr\ targets, it is often very important to 
know precisely, with which type the user is operating, as the limited resources
are important to consider. Substituting the \cc{bool} type with the \cc{int8\_t} type ensures that the size of 
the \cc{bool} variables is known. Also, it can be changed as needed in the 
\mpsid{TypeSizeConfiguration} in each model created with \mb\ separately, see \ref{mbdrmodules}.

The C++ standard explicitly allows the \cc{bool} type to participate in integer promotions. 
This ensures further the compatibility of the custom-written text code, which may use actual 
\cc{bool} type, with the code generated with substitution of \cc{bool} to \cc{int8\_t}, 
as the user \cc{bool} will be promoted.

%Limitations
Among limitations of the approach, when \cc{bool} is translated to \cc{int8\_t}, one can consider \cc{std::vector<bool>}. 
Since the word ``bool'' itself is never generated, it is not possible to use the specialization 
of the template, which ensures storage optimization, through higher processing load when
extracting a value from such a \cc{vector} though.

The \cc{bool} type from \mbdr\ is kept in the \pcpp. This decision is arguable, and can, of-course, be
changed in the future.

\subsection{Modules and C++}
\lsec{modules}

%TODO Reused in MBEDDR CHAPTER

% In C++ as well as in C there is no clear concept of a module. The \mbp\ improves on it, defining 
% modules, \cite{Voelter:MoDELS:2010}. A C module is a concept, from which the header and the .c files are generated in mbeddr.
% Flagging an object (function, variable, structure, etc.) in a module as \cc{exported} causes the declaration of the object to appear
% in the header, and thus the object starts to be usable by other modules.
% 
% Another issue with the C language is the one and only global namespace. The \mbp\ improves on it 
% by introducing so-called name mangling. All names of the module contents are prefixed with the module name,
% when generated to the C text code. Thus the object with the same name but from different modules do
% not cause a name clash.

The \cpppl\ is an improvement over the C language by itself. There is, however, no concept of a module,
which can bring certain advantages and disadvantages. As a disadvantage can be named a potential disorder
in placing implementations for classes and functions, as the language itself does not enforce a clear
one place for them. Potential name clashes are also disadvantageous. 
As an advantage, conditional compilation can be used in C++ to include different files
with different implementations, which is often used to create cross-platform code. Additionally, classes
can be used analogously to modules.

The name clash problem is solved in C++ by introducing the namespaces. The programmer defines a namespace 
and then places there all the exported code. The nodes with equal names from different namespaces do not clash.

C++ however does not elaborate on the disadvantage of having declarations and implementation not organized in 
one place, and being fully disjoint. \mbdr\ compensates on it, \ref{mbdrmodules}.

After comparing the two approaches of \mbdr\ and of the \cpppl\ to improve on C, the hybrid approach
has been picked. Firstly, the C++ implementation modules have been introduced. And secondly, the namespaces
can be defined and used.

In \pcpp\ modules are program modular elements, where a programmer declares and implements C++ classes,
and the .h and .cpp files are automatically generated from them. This eliminates the need to 
create and control two files for each logical program element, like a class, and keep them synchronized with
each other. For example, changing an interface of a class would require at first opening the header file, 
and introducing the changes in the declaration, and then rewriting the corresponding part of the .cpp file 
in strong correspondence with the header. This work decrease in order to change an interface can improve productivity.

Namespaces are introduced in the \pcpp\ to avoid inter-modular name clashes, they work similar to 
the namespaces in C++.

The need in the \cc{using} directive does not emerge the same strong as in plain text editors, because the inputting of namespace names is
made effective within the projectional editor. Long names of namespaces are appended to the end of method
definitions for classes, to be exact about the namespace, but the declarations are kept looking short, with only 
the last namespace shown, \fig{namespaces}.

\ms{namespaces}{Example of Nested Namespaces}

In general, the generator is responsible in generating correct namespace 
resolution, and the projectional C++ programmer can benefit from short and clear namespace
presentations, which are also quite handy to input.


\subsection{Memory Allocation}

In the C programming language memory (de-)allocation on the heap is happening via calls to the dedicated library functions. 
C++ instead introduces separate language construction for memory allocation and de-allocation. 

The \cc{new} and \cc{delete} together with their array versions are simply an expression and a statement correspondingly,
which perform the required operations.

Creating them is relatively easily possible by inheriting from the \mpsid{Expression} and \mpsid{Statement} \rgp{concept},
and introducing the corresponding typing for the \cc{new} expression, as described in \ref{expressionslang}.

When a segment of memory is allocated as an array, it has to be de-allocated as an array either.
As it is stated in the C++ standard (\cite{cpp11}) the behavior of simple de-allocation of an 
memory segment, allocated as an array, results into undefined behavior. A simple check is introduced
in order to control this, see the \rsec{deallocchek}. 


%TODO Check reference


%In C++
%Implicit Assignment from void*
%declare functions before use


%TODO : Exceptions!

%LOL http://stackoverflow.com/questions/3027177/what-are-the-differences-between-c-and-c
%Quote:
%In terms of power:
% - C is a chainsaw.
% - C++ is a 50-foot tall earthmover that mows down everything in its path, and has giant chainsaws sticking out of its wheel hubs to boot.
%Sure, both can clear trees out of your way, but...
%
%In terms of special:
%
% - C is a rockstar.
% - C++ is a narco-syndicate collective of superheroes.



% Let's go back to somewhat less ``intuitive'' comparisons...



%     \begin{enumerate}
%       \item Differences between C and C++
% 	\begin{enumerate}
% 	 \item Reference Type and Boolean Type
% 	 \item Modules Support
% 	 \item Object Oriented Approach Support
% 	\end{enumerate}
%     \end{enumerate}


% Need in C++
% Challenge to implement
% MPS support
% Comprosises - trade of's

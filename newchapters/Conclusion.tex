\declchap{Conclusion}{concl}

Here we briefly review the accomplished contributions and a future work possible.

\declsec{Overview of the Work Performed}{contrib}

In this work we were extending the \mbdr\ C language with programming constructs from the \cpppl, which 
resulted in a software, we call here  \pcpp. 

A new \rg{ide} for C++ has been created, which supports projectional editing for 
a subset of the C++ programming language. The new \rg{ide} has been designed with several goals 
in mind.

First,  \pcpp\ should serve as a modular basis for the future extensions to C++. The 
potentially possible extensions are figured out and proposed as a future 
work.

Second, the experience of the \pcpp\ programming has to be more safe and informative,
when compared to the regular C++ programming. Various pitfalls, usual for C++ programming,
have been explained in this work, and tools have been introduced within the new \rg{ide},
designed to compensate on the pitfalls.

While improving on the C++ programming experience, an application of analyses has been
found to be useful. Various analyses have been applied in \pcpp. The analyses have been
classified orthogonally by the purpose to \rgp{informativeanalysis} and \rgp{preventiveanalysis},
and by the running type to \rgp{analysisondemand} and \rgp{selfrunninganalysis}.


A method to support, to some extent, a code project guidelines, precisely, naming conventions,
has been proposed in this work. The idea to store the project related information together 
with the source code has been formulated, its advantages have been listed.

The projectional approach to create an \rg{ide} has been compared to a traditional textual approach.
The advantages of the projectional approach have been listed. The potential problems while developing,
using and evolving a projectional \rg{ide} and a code produced by it, have been identified.

\pcpp\ as a language has been designed, to represent, in the end, a complete enough subset 
of the \cpppl\ for  \rg{stl} to be recreated in it in the future. The creation of a 
projectional \rg{stl} copy was out of scope for this work, however.
The completeness of the \pcpp\ language was described in this work, and
the future work was proposed, which is going to be needed in order to support \rg{stl}.

Generalized principles of language re-engineering in a projectional \rg{ide} have been
discovered and formulated. The principles can be reused when creating an \rg{ide} for 
a language, to make the language more readable, more expressive, higher cross-platform,
and safer and more convenient to use for a programmer. 

Various language modularity and extensibility problems have been considered on 
practice while developing the \mbdr\ C++ extension. \pcpp\ has been developed in 
such a way, that the \mbdp\ languages do not have a dependency on \pcpp, which
allows for, to a certain degree, separate development of the projects.

In parallel,  while extending \mbdr\ to build a new C++ \rg{ide}, the facilities 
for extensibility, provided by \jbmps\ were researched. Extensibility has been 
analyzed separately for each view on a language. Whenever the degree of extensibility 
was not considered to be high, workarounds and future \jbmps\ improvements were proposed.

The question of building analyses in \jbmps\ for a new language has been discussed.
Three phases of analysis running were identified, and for each phase potential problems
were highlighted. \jbmps\ improvements have been proposed to make the analysis creation 
more productive.

Finally the potential future work has been described, including the ways to get the full C++ language 
support, the need to test the new \rg{ide} on practice, the problem to create an importer for the
textual C++ code and its potential complications have been described.

\declsec{Future Work}{cppcompleteness}

Here we briefly describe the way, \pcpp\ could be developed in the future.

\subsection{Full Language Support}

One of the main target for \pcpp\ in the future has to be developing 
all of the original C++ features. This is needed both for the convenience of 
a \pcpp\ user, and for the ability to import an existing code base, which 
can use all the constructions, possible in C++.

One of the big challenges for the language completeness is a development 
of full templates support.
A very important part of the \cpppl, is undoubtedly  \rg{stl}. In order to acquire
all language capabilities, as it is usual for a C++ programmer, after supporting all features
of the language itself, \rg{stl} has to be implemented in \pcpp. Alternatively , if a powerful
importer is developed first, the \rg{stl} could be potentially imported. 

\subsection{Investigating the Language Use}

In order to continue the \pcpp\ development a user has to be found for the \pcpp, who
will use the language on practice. This will allow to figure out in the fastest way,
which language features left are the most desirable on practice, which new features,
similar to those described in the \rsec{advanced}, could first be developed.

No matter the \cpppl\ is not completely supported by \pcpp, the ability to use the
object-oriented programming paradigm in \mbdr\ represents a qualitative improvement over
the existing C language in \mbdr. This fact represents an advantage, which can
attract any \mbdr\ user towards  \pcpp\ even before the \cpppl\ is entirely supported.

\subsection{Extending Projectional C++}

Following one of the goals to create \pcpp, extensions for it could be created in a 
modular fashion.

As examples of such extensions could be:
\begin{itemize}
 \item Language constructions emulated by preprocessor and templates as in \cite{alexandrescu}.
 \item Classes extensions to support messaging as in Objective-C, or signals and slots as in the Qt project.
 \item Object-oriented design patterns could be researched for the ability to be supported on the 
 language level.
 \item Higher-level models, which would generate to classes together with semantic analyses for them.
\end{itemize}

Other \pcpp\ extensions can be invented, including specific extensions for various domains.


\subsection{Projectional C++ Importer and Other Tools}
An importer for  \pcpp\ could be developed, which would allow to reuse in  \pcpp\ the
existing textual code base. One of the special challenges to develop such an importer would 
be a conversion from regular C++ templates into the \pcpp\ templates with C++ concepts.
A debugger of \mbdr\ could be extended in order to support the \pcpp.


\subsection{JetBrains MPS Evolution}

This work suggests some improvements for \jbmps\ itself, see the \rsec{extensibility} and the 
\rsec{analysesandcomplexity} for example.

If \jbmps\ gets updated, taking some of the mentioned potential improvements into consideration,
 \pcpp\ could be improved benefiting from the new \jbmps\ features.
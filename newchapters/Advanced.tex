\declsec{Advanced IDE Functionality}{advanced}

As the projectional editor works directly with an  \rg{ast}, it is possible to provide some programming on the 
\rg{ast} to improve the user experience. we call this additional programming as an advanced editor functionality, and 
discuss it in this section.

\subsection{Primitive Renaming Refactoring}
Directly from the nature of the projectional editor, without any additional effort, comes a feature to 
perform primitive renaming refactorings\footnote{More complex renaming, like renaming a function \emph{together with} all its overrides,
are not coverred by the default behavior of a projectional editor}.

If a node of an \rg{ast} gets to be referenced somewhere else, it is referenced by the use of its unique internal
identifier. The name of the node is a property of the node, which is not playing any role in
referencing the node from somewhere else. Thus renaming, dislike the way it is performed in text
editors, does not involve replacing the name all around the code. Instead, just the name property 
of a node is changed.

The renaming refactoring comes out of the box, without any additional efforts, thanks to the nature of the
projectional editing.

As an example - renaming a class or a method would mean just changing its name where it is declared first.
No search for usages and multiple replacements are going to be involved.

As a disadvantage here one can think of moving a code from one place to another. For example, moving 
a method from one class to another. The member fields of the source class, even if present in the destination
class, will not immediately fit into the moved code, as it is still referencing not available anymore
identifiers of the source class. This effect is also mentioned in the \rsec{comparison}.

%TODO MPS support for refactorrings!!! - research and at least mention

\subsection{Getter and Setter Generation}
\label{getterandsetter}

In order to provide an access to encapsulated class properties, often expressed as member fields in C++, 
two access functions are usually defined, known as a getter and a setter. The getter is used to read the property,
and the setter is used to set the property to a new value, after checking the validity of the new value.

The job of declaring and prototyping the two functions can be automated with a \jbmps\ \rg{intention}, \fig{getterandsetterintention}.

\msnozoom{getterandsetterintention}{Calling the Generation of Getter and Setter}

The result of the \rg{intention} work is, as expected, two methods declared and prototyped, \fig{getterandsetter}.

\msnozoom{getterandsetter}{Getter and Setter Generated}

The way the editor names the getter and setter can be controlled through the \mpsid{NamingConventions} \rg{concept}, which
is discussed in \ref{namingconventions}.

The getter and setter are declared in the public section of the class, which is automatically created, if
not found there already.

The getter is rather simple, it just returns the value of the member field.
The setter is somewhat more complicated. 

Firstly, it is designed to return a \cc{bool} value by default. This is made
to remind the programmer, to include some checking of the value and return \cc{false}, if the check locates a wrong value passed. 

Secondly, the parameter of the setter is typed appropriately. As C++ by itself imply the performance maximization, the type
of the parameter for the setter depends on the member field type. And when the type represents a composite
structure, like a class, it is passed by a constant reference, instead of value, \fig{compositesetter}.

\ms{compositesetter}{Setter Works with a Constant Reference for Classes}

Passing a composite parameter as a value involves an overhead of allocating the necessary memory and 
then copying the contents in the newly allocated instance. To access the parameter in both cases pointer
arithmetic is still going to follow.

\subsection{Naming Conventions}
\label{namingconventions}

In the C++ development projects some code writing conventions are usually agreed upon. For example, 
in Google coding style guide for C++, \cite{googlecppstyle}, it is stated that all member fields must end up with ``\_'' sign.

For the consistency and uniformity purposes each project has to have some agreements on the way the code is composed.
They can include code formatting rules and naming rules. 

To some extent the formatting conventions are fixed by the way the projectional editor shows the 
\rg{ast}, see, for example, \ref{visibilitysections}.

The naming rules, however, should be additionally controlled. The naming rules, accepted in a project, we call 
naming conventions here.

In order to perform the naming conventions controll a new \rg{concept} has been introduced, called \mpsid{CppNaming Conventions}, 
\fig{namingconventions}. It is created once per \jbmps\ \rg{model}.

\ms{namingconventions}{C++ Naming Conventions Concept}

In the \mpsid{CppNamingConventions} one can give a standard prefixes for getters and setters, which are used during the code 
generation, see \ref{getterandsetter}. The argument prefix is used in the setter generation, to name the
setter argument.

The member prefix is used on the member fields to check their naming. It possible to change the prefix and the editor will control
each of the member field names to comply.

Whenever a field is not named in a proper way, an error \rg{intention} is available, see \ref{intentions}, 
which automatically renames the field, \fig{renameconventionally}.

\msnozoom{renameconventionally}{Intension to Rename a Field - Naming Conventions}

The \mpsid{CppNamingConventions} \rg{concept} was created to demonstrate some new possibilities, which can be added to the projectional C++.
In this case the naming conventions can be seen as a part of project configuration, in a very common sense. This is generalized in the
\rsec{genprinciples}.

Some remarks on the \mpsid{CppNamingConventions} \rg{concept} are given additionally in \ref{remarksonnamingconventions}.


\subsection{Method Implemented Check}
\label{implementedcheck}

When using the \cpppl\ it is possible to declare a class in one file, and then implement its methods in other, potentially
several, files. As there is no clear concept of a module in the language itself (more on this in the \rsec{modules}) usually 
different coding standards have to improve on it, requiring, for example in the Qt Project \cite{qtcodingguide}, one class to be 
declared in a the-same-named header file, and the implementation for it to be in one similar named .cpp file.

As the language itself does not argue about the place for methods, \rgp{ide} usually do not check if the method is implemented.
Neither do compilers. And only at the linking stage it can appear, that the linker is not able to locate the method. The error
message from the linker can be rather obscure, as the linker operates on a much lover abstraction level, rather than the programmer 
using the C++ language. 

After introducing the new module \rg{concept} as in the \rsec{modules}, it is possible to check each method and find out if they all 
are implemented on the very early coding stage.

A check is introduced in the \pcpp, which respects inheritance chains, with overloaded functions and \rg{purevirtual} functions,
and creates a warning for the programmer hinting that a certain method has not been implemented. Together with the class declaration, 
generating the assignment operator declaration and a copy constructor, the method implemented check makes the programmer to 
fully define the class together with the most important methods, affecting the memory operations behind the class instances life-cycle.


\subsection{Abstract Class Construction Check}
\label{preventiveabstract}
% Check for abstract classes construction

In this work we have already defined the notion of \rg{abstract} classes in C++ and some related to them questions in \ref{abstractclasses} and
in \ref{cpppolydefs}. Mainly the \rg{acsa} and the problem of determining \rg{abstract} classes were discussed.


\msnozoom{abstractinstance}{The Abstract Class Construction Check Signaling an Error}

An opened question which stayed, however, is providing some help to the programmer in the usage of such classes. Namely, any instantiation
of an \rg{abstract} class has to be forbidden. The \rg{abstract} class as a type, however, should stay, and the pointer type to it should not
be anyhow affected by the fact that the class is \rg{abstract}.

The decision was taken to check the places, where an instance of an \rg{abstract} class can appear, and deprecate such situations, marking the 
corresponding node as error nodes.
\jbmps\ supports so-called non-type-system checks, which come useful in this situation. Any node on the \rg{ast} can be analyzed, and 
if the error happens to be detected, there is a chance to associate it with the node and a textual error message, which will
clarify the programmer, what exactly went wrong. 

Among the places, where a class can be instantiated are variable declarations field declarations, method and function arguments.
The return type of the function can imply an \rg{abstract} class instantiation. For each of such places a check is performed, 
and if the class type is detected, it is checked against being \rg{abstract}. Error message appears to indicate the problem, 
if found, \fig{abstractinstance}. 

The procedure of determining if a class is \rg{abstract}, can happen every time, when the check is performed. These are all 
variable declarations, function and method arguments and return types, and memory allocations. The amount of computational
load can be significant. The question of this complexity is discussed in the \rsec{analysesandcomplexity}.


\subsection{Array Deallocation Check}
\lsec{deallocchek}
% When it is easily possible to see an array allocated, then array deallocation is requested
Whenever some allocated on the heap data is about to be deallocated, in C++ two operators can
be involved, \cc{delete} or \cc{delete[]}. The first one is dedicated to work with memory, allocated
for a single object be he operator \cc{new}, the second deallocates memory, allocated for an array
of object by the operator \cc{new[]}.

The C++ standard does no specify any concrete behavior for wrong deallocation of an array with 
an operator \cc{delete} (without square brackets), describing it as undefined, \cite{cpp11}.
 
A simple check is performed, and array dealocation is marked, when is performed wrongly. 

Of-course, a complete check for this can only be performed by a complex analysis of the data-flow.
This is left for the future work.


\subsection{Analyses to Implement as a Next Step}

The advanced \rg{ide} fuctionality can be grown incrementally. 

Among the necessary checks to increase the \pcpp\ quality one can find:
\begin{itemize}
 \item Detection of virtual class
 \item Calculation of class-sizes
 \item A check for virtual destructor in a virtual class
 \item A check for a default contructor when an array of obejcts is created
 \item A check that an exception thrown gets caught eventually
\end{itemize}

And indeed many more. Such checks are left for the future work.

%TODO Default constructable and arrays! Require constructor by default

%TODO Check for Virtual Destructor 
\declchap{Technologies in Use}{techno}

The C++ programming language developed through out this \MT\ is based on two technologies, which are 
introduced in this chapter. The first technology is the \jbmps\ language engineering environment, which 
provides the core foundations and means for incremental language construction. The second technology is
the \mbdp. 

\declsec{Jetbains MPS}{mps}


\declsec{MBEDDR Project}{mbeddr}

The \mbdp\ is a software built with the use of \jbmps.

The \mbdrp\ represent mainly an implementation of the C programming language in the \jbmps\ environment. Having embedded systems
and software for them as the main focus, \mbdr\ provides certain language extensions to empower the programmer in the mentioned domain \cite{mbeddr-wave}, 
\cite{Voelter:MoDELS:2010}. 

Being a different language the \cpppl\ shares a lot of commonality with the C programming language. 
As \jbmps\ allows, to some extent, see \ref{modularity}, incremental language construction, the \mbdrp\ represents 
a suitable basis for the \cpppl\ implementation in \jbmps\ . 

The use of \mbdr\, however, could not be purely incremental, and required some changes to the \mbdr\ itself. 
The changes were introduced however, in a way to make \mbdr\ simply more extensible in general, and not by
adopting it to the current work needs.

In this section I describe the \mbdp, as the current work is based on it.

No matter the \mbdp\ has  (one) C language with extensions as an outcome, internally, as a \jbmps\ software
it is represented as several \jbmps\ languages. In \jbmps\ a language corresponds to a module.

All \mbdr\ languages are named starting from \mpslang{com.mbeddr.} name part. In this document I usually ommit it,
keeping the last word of the name only. E.g. \mpslang{com.mbeddr.expressions} is called simply \mpslang{expressions} here.

\subsection{\mbdr\ Expressions Language}

The \mpslang{expressions} language contains definitions for all expressions, possible in the \mbdr\ C language.
As in object oriented programming languages \rgp{concept} of the \cc{expressions} language form inheritance hierarchies. 
\jbmps\ is capable of showing a given \rg{concept} in a hierarchy. 

\msnozoom{minushierarchy}{\Rg{concept} Hierarchy Example}

The \fig{minushierarchy} shows a hierarchy for the \mpsid{MinusExpression} \rg{concept}. In a similar way all expressions of the 
C programming language are implemented in the \mpslang{expressions} language.

Whenever there is a need in the \cpppl\ to extend the C programming language with a new expression kind, like 
object member reference, \mpslang{new} expression and so on, a point of inheritance has to be found in the 
\mbdr\ \mpslang{expressions} language to base a new \rg{concept} on it.

Additionally, the \mpslang{expressions} language defines  C language types. 

\msnozoom{typehierarchy}{\mbdr\ Type Hierarchy Example}

All \rgp{concept} corresponding to C types are based (directly or indirectly) on the \mpsid{Type} \rg{concept}.
For example, the hierarchy of \mpsid{IntType} \rg{concept} is demonstrated in \fig{typehierarchy}. 

In order to add a type to \mbdr\ C language, one should inherit from the \mpsid{Type} \rg{concept} as well.
Such inheritance automatically allows the new type to appear at all places, where a type in general can be found
in the C language or its extensions.


\subsection{\mbdr\ Statements Language}

\mbdr\ \mpslang{statements} language contains definitions for C language statements. The \mpsid{Statement} \rg{concept} serves as the
base for inheritance, and represent by itself an empty line, or no-statement.

In order to create a new statement, like \cc{delete} statement in C++, the inheritance should start from the \mpsid{Statement} \rg{concept}.

\ms{stexprmarked}{Example of Multiple Languages Used Together}

The \mpslang{statements} language actively uses the \mpslang{expressions} language, \fig{stexprmarked}. In the \mbdr\ code snippet
the nodes coming from \mpslang{statements} language are marked green, and the nodes, coming from \mpslang{expressions} language are
marked yellow\footnote{not marked with color is an instance of the \mpslang{Function} concept, which comes from the \mpslang{modules} language}. As the example shows, \mpsid{if} statement and \mpsid{return} statement are coming from the \mpslang{statements} language,
but inside they contain as children expressions. This is an example of language modularity in \jbmps, used by \mbdr.

% TODO continue here about modules language








\declsec{The C++ Language}{cpplangitself}

This work is basically an implementation of the \cpppl\ in \jbmps. No matter, the whole work 
is about the C++ implementation, it is infeasible to describe the language in detail itself
here. Thus we give a references to literature about the language in this chapter only. Certain 
aspects of C++ are explained in depth during the description of the C++ implementation in \jbmps\ itself in 
the \rchap{cpp}.

Various literature is available, to get acquainted with the language closer. The most complete
guide to the language, covering \rg{stl} as well is \cite{stroustrupcpp2000}. The more 
easy-to-read and more suitable for beginners book on C++ is \cite{schildtcpp}. A newer book,
oriented towards intermediate level C++ programmers and updated to the recent C++11 standard, \cite{cpp11}, and describing 
\rg{stl} as well, \cite{pratacpp}.

The collection of techniques to get more effective C++ programs is \cite{meyerseffcpp}. Templates are explained in 
detail in \cite{josuttistemplates}. The book dedicated to ad hoc polymorphism and advanced template meta programming can be 
also of interest as an approach to language engineering in a certain sense, \cite{alexandrescumeta}.

The \cpppl\ is a mature language, with long traditions, and high flexibility, \cite{alexandrescu} can serve as an example. 
It will not be possible to simplify language, removing features from it, which will restrict the language use, 
most importantly the \rg{stl} support. Thus in this work we try to research, how the editor can be more supportive 
for the user of the \emph{existing} complex language, to eliminate usual mistakes made while programming, as 
well as provide help in structuring the code.
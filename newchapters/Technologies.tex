\declchap{Technologies in Use}{techno}

The C++ programming language, developed through out this \MT, is based on two technologies, which are 
introduced in this chapter. The first technology is the \jbmps\ language engineering environment, which 
provides core foundations and means for incremental language construction. The second technology is
the \mbdp. These technologies are discussed in this chapter.

\declsec{Jetbains MPS}{mps}

\jbmps\ stands for JetBrains Meta Programming System. In this \Rg{ide}-like software it is possible to develop \Rgp{dsl}.
The approach taken in the \jbmps\ is rather unique, and it is considered to be advantageous in many ways, \cite{Voelter:MoDELS:2010}.

% No way to extend mbeddr otherwise
% Thus \jbmps\ is a good suitable technology for the practical implementation of the \MT\ goal.

In general, the way the language is describe in \jbmps\ corresponds to the way, described in the \rsec{modularlang}.
Here we will describe the process in practical details, as it is crucial for understanding the practical part of this \MT.

In this section we will go through a definition of one \rg{concept}, describing the facilities, \jbmps\ provides to support
it. Each \rg{concept} is described by several views on it. 

\msnozoom{mpsif}{\jbmps\ User Interface, \cc{if} Statement \Rg{concept}}

Among such views there are \rg{concept} declaration (or language structure) view, editor view, behavior view, constraints view, type system view,
generator view and few more views. As a language consists mostly of the included in it concepts, 
the whole language is presented by the mentioned views as well, where each view on the language contains all views of the kind on each language \rg{concept}. 
Left part of the \fig{mpsif} demonstrates the views on the \mpslang{statements} language.

% The representation of a new language created in concepts and views to them, present a seamless approach to creating 
% a new language with a projectional editor for it.

The \mbp\ is a software, separate from \jbmps\, but based on it, representing an extensible C language implementation with extensions.
we will use \rgp{concept} from the \mbp' throughout this section as examples to demonstrate MPS. Different C language parts are going to be 
decomposed into \rgp{concept} and these concepts are going to be defined using \jbmps. The reader should not confuse though the \mbp\ and 
\jbmps\ itself: the former is a software, developed in the latter and is used to demonstrate the latter.

Interestingly enough, \rgp{dsl} are used in order to define new languages. Each view features a language used to describe a new \rg{concept}.

\subsection{Concept Declaration}
\label{mpsconceptdeclaration}

\Rgp{concept}, as it is described in the \rsec{modularlang} represent a class-like types for nodes of an \rg{ast}. This terminology is kept in \jbmps\ and 
MPS concept has the same meaning as \rg{concept} term used in the \rsec{modularlang}. we use the term ``concept'' both in general, to describe an \rg{ast} node type,
and in particular referring to an MPS concept.

% Repeats \rsec{modularlang}
%
% One defines a language in it not through the canonical grammar approach, but instead through defining so called concepts, and relationships 
% between them similar to those in ER diagrams. A logical part of a language can be made a concept. 
% Related to C++, class, method declaration, field declaration can be represented as a concept. 

The \fig{mpsif} on the right part demonstrates a declaration of \mpsid{IfStatement} concept from the \mbdr\ \mpslang{statements} language.
It corresponds to the \cc{if} statement of the C language.

At first, the \rg{concept} is named, and a \rg{baseconcept} is defined. The \mpsid{IfStatment} \rg{concept} inherits from the \mpsid{Statement} \rg{concept}.
This allows the \mpsid{IfStatement} to be used at any place at which the \mpsid{Statement} could be used, and inherits like in object oriented programming
all data and behavior of the \mpsid{Statement}.

Next, it is defined, which interfaces the \rg{concept} is going to implement. For example, by implementing \mpsid{ISteppableContext}, the \mpsid{IfStatement}
supports the \mbdr\ debugger when stepping in the body of the \mpsid{IfStatement}.

The ``instance can be root'' property defines, if it is meaningful, to create a \rg{concept} without a parent \rg{concept} for it. In the case
of the \mpsid{IfStatment} it does not make sense, as the statement should belong to some block. The \cc{true} value can be used, e.g. for modules,
as they do not have any outer concepts, and can be seen as a document in \jbmps.

The ``properties'' part defines if the described \rg{concept} instances should have some primitive type data fields (string, boolean, int).
An example of a property could be a \mpsid{name} property of a variable declaration. The \mpsid{IfStatement} \rg{concept} does not specify any 
properties, neither does it inherit any from the \mpsid{Statement} concept.

The ``children'' section describe which nodes can be children on the \rg{ast} of the \mpsid{IfStatement}. Each child is assigned with a role and cardinality. 
For example, the \mpsid{IfStatement} should have exactly one child of \rg{concept} \mpsid{Expression}, it has a role ``condition''.

The ``references'' section describes in the similar way as in the ``children'' section, which nodes could be referenced by the node of a given \rg{concept}.
Referencing can be used, to bind a given node, to a node, located somewhere else on the \rg{ast}. As an example, a variable usage in expression shall reference
the variable declaration, to express precisely, which variable is used.

Finally, some attributes of a \rg{concept} follow, which do not have a primary importance for the discussion here. The ``alias'' is used to 
name a \rg{concept} in a short way, to allow for quick instantiation in the editor. The ``short description'' is shown to hint a programmer on the
alias meaning. A \rg{concept} can be made abstract in the ``concept properties'' section. Abstract \rgp{concept} are purely used in inheritance 
to create other non-abstract concepts with a common parent.

\jbmps\ separately allows to define so-called \rgp{interfaceconcept}. Interface concepts are \rgp{concept} which can not be instantiated, but 
which serve as a base for inheritance and implementing and interfaces as Java classes do. A \rg{concept} can have only one base concept, but can
inherit from/implement many interface concepts.

\subsection{Editor View}
\label{mpseditor}

The editor view, \fig{mpsifeditor}, is designed to give a look for a node of a \rg{concept}, and a way to input it. This is where the projection of an \rg{ast} node is defined. 
As editors are mostly defined to look like text, a program in the \cpppl implemented in MPS looks almost like a regular C++ code.

In the editor view one defines an editor for a \rg{concept}. \jbmps\ introduces here a bit confusing terminology, calling one editor for a given
\rg{concept} concept an ``editor concept''. Thus an error message ``An editor concept not found for a concept X'' would mean that no editor
has been defined for the \rg{concept} \mpsid{X} in the editor view for it. In this work we call the content in the editor view for a \rg{concept} \mpsid{X} an 
``editor for the \rg{concept} \mpsid{X}''. The same terminology applies to all the following views related to a single \rg{concept}.

\msnozoom{mpsifeditor}{Editor View for the \mpsid{IfStatement} \Rg{concept}}

The editor for a \rg{concept} defines the visual representation for a node of the concept, using special syntax. For example, \fig{mpsifeditor} defines
and editor for the \mpsid{IfStatement}. At first the ``constant'' non-changeable by a programmer text is given, which is ``if'' and ``(''. Then
the child \mpsid{condition} is referenced, so that after the ``if('' the user will be able to input an expression for the condition of the \cc{if} statement.
The editor can be configured to show or hide some parts of a node, depending on some condition, e.g. hiding \cc{else} part of the \cc{if} statement, if it 
is not defined anyhow. Editor are also responsible for all the interaction, a user experiences when editing a code in \jbmps.

\subsection{Behavior View}
The behavior view, can be used to define certain methods for a concept. A concept is represented there similar to a Java class, and
it is possible to define the methods in a Java-like language. \Rg{concept} inheritance is taken into account like in Java. 

A \rg{concept} constructor can be defined there to initialize by default a newly created node of a \rg{concept}.

\msnozoom{mpslocvarrefbeh}{Behavior View for the \mpsid{LocalVarRef} \Rg{concept}}

The \fig{mpslocvarrefbeh} shows the behavior of the \mpsid{LocalVarRef} \rg{concept}. This \rg{concept} represents in the
\mbp\ C language an expression, referencing a local variable. The local variable declaration is stored as a reference \mpsid{var}
in the \rg{concept} nodes.

Two convenience methods are defined for the \mpsid{LocalVarRef} \rg{concept}, to get an easy access to the local variable properties.


\subsection{Constraints View}
The constraints view can be used, to limit in a desirable way \rg{concept} property values and relationships, a node of the \rg{concept} can 
have to other nodes. It is possible to take into account any sort of context, and thus create a context-aware/sensitive \rg{concept}.

\msnozoom{mpsidnameconstr}{Editor View for the \mpsid{IfStatement} \Rg{concept}}

The \fig{mpsidnameconstr} shows constraints defined for a property \mpsid{name} of the \mpsid{IIdentifierNamedConcept} \rg{interfaceconcept}.
All \rgp{concept} which have a name, which must confirm to the identifier naming restrictions, can implement the \mpsid{IIdentifierNamedConcept},
to immediately get the desired characteristic.

The \mpsid{name} property is programmatically restricted by the use of Java-like code snippet, \fig{mpsidnameconstr}. In a similar way
relationships to children and referents can be restricted.

It is possible to create a pure Java class withing an \jbmps\ language, and use it in almost any \rg{concept} view in the \jbmps. 
In the \mpsid{IIdentifierNamedConcept} \rg{interfaceconcept} constraints the \cc{CIdentifierHelper} class was used to check the
name property on collision with the C language keywords.

Constraints play an important role for the editor work. A programmer is presented with a list of choices, when inputting
a new node on an \rg{ast}. The choice of nodes is \emph{defined} with the constraints.

\subsection{Type System View}
\label{mpsts}
\jbmps\ has a special support for creation of typed languages. Types are mainly used in expressions. An expressions may count on 
a certain sub-expression to have a given type, or a type, compatible with it. Whenever the expectation does not meet the reality,
a warning or error can be displayed to a programmer. 

\msnozoom{mpsternaryts}{Type System View for the \mpsid{TernaryExpression} \Rg{concept}}

A \rg{concept} representing nodes which can have type in \mbdr\ inherits from \mpsid{ITyped} interface concept. In the type system view
certain rules must be defined with the use of a special language, which define the type or type comparison rules for a \rg{concept}.
Moreover, the type system language can be used to infer a type for a given node, \fig{mpsternaryts}.

The \fig{mpsternaryts}, demonstrates the use of the \jbmps\ type system language to infer a type of the \mpsid{TernaryExpression} \Rg{concept}
node. The syntax is on the one hand self-explanatory when reading, but on the other hand could be rather confusing when crafting.

When none of the existing type system rules can resolve a given typing problem, the so-called replacement
rule can be invoked in \jbmps. Replacement rules are defined for a given \rg{concept}. A snippet of Java-like
code must be given to explicitly analyze a node and assign a type to it.


\subsection{Non-Type-System Checks}
\label{mpsnontschecks}

Additional to type system and constraints limitations on an \rg{ast} structure can be put via implementing 
non-type-system rules. In this work we often call them simply ``checks''.

For a given \rg{concept} a code snippet can be given, to check all nodes of the \rg{concept}. After the 
check is performed, there is a chance to mark a node as an error node, or create a warning on the node.
It is possible to provide some textual hint for a language user, to identify the reason of the error or warning, 
see the \fig{abstractinstance} for an example.

Non-type-system rules are used in this work to create \rgp{preventiveanalysis}, and are discussed additionally in \ref{reportinganalysis}.

Extensibility for the checks is evaluated as an extensibility for analyses, \ref{analysesext}.

\subsection{TextGen View}

To make use of the code in the projectional editor, further tools must be invoked on it, e.g. parser, compiler, etc.
Normally they work with a textual representation of the same code. In order to obtain the textual code from the \rg{ast} in the projectional
editor generators are invoked. 

Generators can be of two kinds. The kind of generator is dedicated to transform an \rg{ast} in one \rg{dsl} into
another \rg{ast} represented in a, usually, lower-level language. The second kind of generators is dedicated to transform an \rg{ast} into text.
Such generators are called ``\rgp{textgen}'' in \jbmps.

\msnozoom{mpsbinexptextgen}{TextGen View for the \mpsid{BinaryExpression} \Rg{concept}}

The \fig{mpsbinexptextgen} demonstrates how a node of a \mpsid{BinaryExpression} \Rg{concept} is converted to text. It is noteworthy 
to say, that when rendering to text the \mpsid{left} and \mpsid{right} sub-expressions, the corresponding \rgp{textgen} are invoked,
making the text generation recursive.


\subsection{Generator View}
\label{generators}
The generator view is one of the most complex \jbmps\ views. A language engineer uses this view to define, how an \rg{ast} composed
in one language, has to be transformed into an \rg{ast} in another language in \jbmps.

In this work we mostly use \Rgp{textgen} to produce a textual C++ code, when the programming in the projectional editor has completed.
Thus, generator view does not have a strong connection to the present work. We do not describe it in details here.

\subsection{Intentions}
\label{intentions}

\jbmps\ \rgp{intention} are special procedures which can be used for automatic manipulations on the \rg{ast} with a node of a given \rg{concept}.

\msnozoom{mpsconstintention}{Toggle Const Property for a \mpsid{Type} \Rg{intention}}

The \fig{mpsconstintention} demonstrates an intention, used to modify the \cc{const} property of a given type. 

For an \rg{intention} to be defined, one has to name it, specify a \rg{concept}, to which the \rg{intention} is applicable,
provide a textual description for it, and finally specify the desired effect. 

There is also a special kind of \rgp{intention}, called ``error intentions''. They are not anyhow fundamentally different 
from the usual \rgp{intention}, except their special purpose to fix an error, when one occur. 

Error \rgp{intention} are visualized using a red bulb icon in \jbmps.

Intentions are accessible in the projectional editor from a context menu, when focused on a target node, \fig{classintentions}. They represent 
a useful mechanism to support a programmer with various automations, including automatic code generation.

\msnozoom{classintentions}{Intentions Available for the \mpsid{Class} \Rg{concept}}

\subsection{Other MPS Instruments}

\jbmps\ provides other instruments to enhance languages. 

\Rgp{action} are used to automate node deletions or editing. For example, deleting an array indexing expression, could be made
to provide a substitution, an array expression itself. Such behavior may seem more natural to a programmer, used to text editing.

\jbmps\ provides a special support for refactorings. Special code snippets in the Java-like language can automate routine operations.
As an example, factoring out a local variable from an expression can be taken.

Additionally, it is possible to extend the functionality of an editor under construction via creating special \jbmps\ plug-ins. 
It can be in particular useful to implement some complex \rgp{analysisondemand}.

\declsec{MBEDDR Project}{mbeddr}

The \mbdp\ is a software built with the use of \jbmps.

The \mbdrp\ represent mainly an implementation of the C programming language in the \jbmps\ environment. Having embedded systems
and software for them as the main focus, \mbdr\ provides certain language extensions to empower the programmer in the mentioned domain \cite{mbeddr-wave}, 
\cite{Voelter:MoDELS:2010}. 

Being a different language the \cpppl\ shares a lot of commonality with the C programming language. 
As \jbmps\ allows, to some extent, see \ref{modularity}, incremental language construction, the \mbdrp\ represents 
a suitable basis for the \cpppl\ implementation in \jbmps\ . 

The use of \mbdr\, however, could not be purely incremental, and required some changes to the \mbdr\ itself. 
The changes were introduced however, in a way to make \mbdr\ simply more extensible in general, and not by
adopting it to the current work needs.

In this section I describe the \mbdp, as the current work is based on it.

No matter the \mbdp\ has  (one) C language with extensions as an outcome, internally, as a \jbmps\ software
it is represented as several \jbmps\ languages. In \jbmps\ a language corresponds to a module.

All \mbdr\ languages are named starting from \mpslang{com.mbeddr.} name part. In this document I usually ommit it,
keeping the last word of the name only. E.g. \mpslang{com.mbeddr.expressions} is called simply \mpslang{expressions} here.

\subsection{mbeddr Expressions Language}
\label{expressionslang}

The \mpslang{expressions} language contains definitions for all expressions, possible in the \mbdr\ C language.
As in object oriented programming languages \rgp{concept} of the \cc{expressions} language form inheritance hierarchies. 
\jbmps\ is capable of showing a given \rg{concept} in a hierarchy. 

\msnozoom{minushierarchy}{\Rg{concept} Hierarchy Example}

The \fig{minushierarchy} shows a hierarchy for the \mpsid{MinusExpression} \rg{concept}. In a similar way all expressions of the 
C programming language are implemented in the \mpslang{expressions} language.

Whenever there is a need in the \cpppl\ to extend the C programming language with a new expression kind, like 
object member reference, \mpslang{new} expression and so on, a point of inheritance has to be found in the 
\mbdr\ \mpslang{expressions} language to base a new \rg{concept} on it.

Additionally, the \mpslang{expressions} language defines  C language types. 

\msnozoom{typehierarchy}{\mbdr\ Type Hierarchy Example}

All \rgp{concept} corresponding to C types are based (directly or indirectly) on the \mpsid{Type} \rg{concept}.
For example, the hierarchy of \mpsid{IntType} \rg{concept} is demonstrated in \fig{typehierarchy}. 

In order to add a type to \mbdr\ C language, one should inherit from the \mpsid{Type} \rg{concept} as well.
Such inheritance automatically allows the new type to appear at all places, where a type in general can be found
in the C language or its extensions.


\subsection{mbeddr Statements Language}

\mbdr\ \mpslang{statements} language contains definitions for C language statements. The \mpsid{Statement} \rg{concept} serves as the
base for inheritance, and represent by itself an empty line, or no-statement.

In order to create a new statement, like \cc{delete} statement in C++, the inheritance should start from the \mpsid{Statement} \rg{concept}.

\ms{stexprmarked}{Example of Multiple Languages Used Together}

The \mpslang{statements} language actively uses the \mpslang{expressions} language, \fig{stexprmarked}. In the \mbdr\ code snippet
the nodes coming from \mpslang{statements} language are marked green, and the nodes, coming from \mpslang{expressions} language are
marked yellow\footnote{not marked with color is an instance of the \mpslang{Function} concept, which comes from the \mpslang{modules} language}. As the example shows, \mpsid{if} statement and \mpsid{return} statement are coming from the \mpslang{statements} language,
but inside they contain as children expressions. This is an example of language modularity in \jbmps, used by \mbdr.

\subsection{Modules in mbeddr}
\label{mbdrmodules}

In C (and in C++ as well) there is no clear concept of a module. The \mbp\ improves on it, defining 
modules, \cite{Voelter:MoDELS:2010}. A C module is a \rg{concept}, from which the header and the .c files are generated in mbeddr.
Flagging an object (function, variable, structure, etc.) in a module as \cc{exported} causes the declaration of the object to appear
in the header, and thus the object starts to be accessible by other modules.

An issue with the C programming language is that there is one and only global namespace. The \mbp\ improves on it 
by introducing so-called name mangling. All names of the module contents are prefixed with the module name,
when generated to the C text code. Thus the object with the same name but from different modules do
not cause a name clash.

The implementation of modules in \mbdr\ can be found in the \mpslang{modules} language. Functions are described there as well,
for the \mpsid{Function} concept example, see the \fig{stexprmarked}, not marked with colors part.

Modules are further included into \jbmps\ \rgp{model}, which correspond to one single \jbmps\ file unit.
Each model should have a node of \mpsid{TypeSizeConfiguration} and \mpsid{BuildConfiguration} \rgp{concept}.
In the former, the sizes for types must be given, in the latter, all directives needed to compile the 
generated from the \rg{model} text.

\subsection{Pointers and Arrays in mbeddr}

In the \mpslang{pointers} language array access and pointer dereferencing expressions are defined as \rgp{concept}.
Similar syntax to them have overloaded operators with classes. Thus this language could also be extended by the \pcpp,
if some reuse is possible there as well. This is discussed in detail in the \rsec{operatoroverloading}.











\declsec{C++ Language}{cpplangitself}

This work is basically an implementation of the \cpppl\ in \jbmps. No matter, the whole work 
is about the C++ implementation, it is infeasible to describe in detail the language 
here. Thus, we only give references to literature about the language in this chapter. Certain 
aspects of C++ are explained in depth during the description of the C++ implementation in \jbmps\ itself in 
the \rchap{cpp}.

Various literature is available, to get acquainted with the language closer. The most complete
guide to the language, covering \rg{stl} as well is \cite{stroustrupcpp2000}. A more 
easy-to-read and more suitable for beginners book on C++ is \cite{schildtcpp}. A newer book,
oriented towards intermediate level C++ programmers and updated to the recent C++ 11 standard (referenced as \cite{cpp11}) and describing 
\rg{stl} as well is \cite{pratacpp}.

A collection of techniques to get more effective C++ programs is gathered in \cite{meyerseffcpp}. Templates are explained in 
detail in \cite{josuttistemplates}. A book dedicated to ad hoc polymorphism and advanced template meta programming can be 
also of interest as an approach to language engineering on top of C++ in a certain sense, \cite{alexandrescu}.

The \cpppl\ is a mature language, with long traditions, and high flexibility, \cite{alexandrescu} can serve as an example. 
It would not be possible to simplify the language, removing features from it, which would restrict the language use, 
most importantly the \rg{stl} support. Thus in this work we try to research, how the editor can be more supportive 
for a user of the \emph{existing} complex language, how to eliminate usual mistakes made while programming, as 
well as how to provide help in a code structuring.